\section{Kết luận}
    \hspace*{0.8cm}Kiến trúc ứng dụng Android được thiết kế theo hướng mô-đun, phân tầng rõ ràng nhằm đảm bảo tính linh hoạt, dễ mở rộng và bảo mật cao. Ứng dụng Android chạy trên nền tảng hệ điều hành Android, vốn được xây dựng dựa trên nhân Linux – nơi chịu trách nhiệm quản lý phần cứng, tiến trình và bộ nhớ. Trên nền đó, Android triển khai nhiều cơ chế bảo mật như sandbox, UID riêng biệt, máy ảo (Dalvik hoặc ART), hệ thống phân quyền truy cập và SELinux để cô lập các ứng dụng, đảm bảo rằng mỗi ứng dụng chỉ hoạt động trong phạm vi tài nguyên được phép. Kiến trúc ứng dụng bao gồm nhiều thành phần như Activity, Service, BroadcastReceiver và ContentProvider, được khai báo trong file AndroidManifest.xml và quản lý bởi hệ thống Android Framework. Dữ liệu và mã nguồn được đóng gói trong tệp APK, trong đó có chứa bytecode .dex, tài nguyên biên dịch và thông tin cấu hình. Ngoài ra, Android còn hỗ trợ các công cụ hiện đại như Gradle, D8/R8, và quy trình desugaring để tương thích với các tính năng Java mới. Nhờ sự kết hợp giữa nền tảng Linux ổn định, mô hình máy ảo hiệu quả, cùng các lớp bảo mật chặt chẽ và công cụ phát triển mạnh mẽ, kiến trúc ứng dụng Android ngày càng hoàn thiện, tạo điều kiện thuận lợi cho lập trình viên xây dựng các ứng dụng hiệu suất cao và an toàn cho người dùng.\\

