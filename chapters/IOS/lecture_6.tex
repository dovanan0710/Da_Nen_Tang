\section{Xử lý Dữ liệu}

\hspace*{0.8cm}Quản lý dữ liệu là một phần quan trọng trong phát triển ứng dụng iOS. Apple cung cấp nhiều giải pháp khác nhau để lưu trữ và truy xuất dữ liệu, phù hợp với các mục đích và quy mô sử dụng khác nhau. Dưới đây là so sánh và mô tả chi tiết các giải pháp lưu trữ dữ liệu phổ biến trong hệ sinh thái iOS.


\subsection{Core Data}
Core Data\cite{Core-Data} là một framework của Apple được thiết kế để quản lý graph đối tượng và vòng đời dữ liệu trong ứng dụng iOS, không phải là một cơ sở dữ liệu thuần túy. Nó cung cấp một cách tiếp cận mạnh mẽ để làm việc với dữ liệu có cấu trúc phức tạp và lưu trữ trong các ứng dụng iOS.

\textbf{Kiến trúc Core Data} bao gồm một số thành phần quan trọng. Đầu tiên là \textbf{NSManagedObjectModel}, mô tả cấu trúc của các entities, attributes và relationships trong ứng dụng. \textbf{NSManagedObjectContext} là không gian làm việc để thực hiện các thao tác CRUD (Create, Read, Update, Delete), trong khi \textbf{NSPersistentStoreCoordinator} điều phối giữa object model và persistent store. Đối với các ứng dụng từ iOS 10 trở lên, \textbf{NSPersistentContainer} cung cấp một cách dễ dàng để đóng gói toàn bộ stack Core Data, giúp đơn giản hóa việc cấu hình và quản lý dữ liệu. Để truy vấn dữ liệu, \textbf{NSFetchRequest} được sử dụng để lấy dữ liệu từ persistent store.

Core Data cung cấp một số \textbf{đặc điểm chính}, bao gồm khả năng quản lý object graph và các mối quan hệ phức tạp, lưu trữ dữ liệu thông qua SQLite backend, và hỗ trợ tính năng tải dữ liệu một cách lười biếng (lazy loading). Hệ thống còn tích hợp kiểm tra dữ liệu và hỗ trợ migration/versioning, giúp dễ dàng cập nhật và bảo trì dữ liệu qua các phiên bản ứng dụng khác nhau.

Về \textbf{ưu điểm}, Core Data tích hợp chặt chẽ với hệ sinh thái của Apple, bao gồm các công cụ thiết kế mô hình dữ liệu trực quan. Nó còn hỗ trợ đồng bộ hóa dữ liệu với CloudKit và có khả năng quản lý bộ nhớ thông minh, tối ưu hóa việc sử dụng tài nguyên.
Tuy nhiên, Core Data cũng có \textbf{nhược điểm}. Đường cong học tập của nó khá dốc, khiến người mới làm quen cảm thấy khó khăn trong việc sử dụng. Việc debug cũng trở nên phức tạp và khó khăn, và Core Data không hoàn toàn thread-safe, điều này có thể gây ra các vấn đề khi thao tác với dữ liệu từ nhiều luồng cùng lúc.

Cuối cùng, \textbf{Core Data} là sự lựa chọn lý tưởng cho các ứng dụng có dữ liệu phức tạp và nhiều quan hệ. Nó cũng rất phù hợp cho các dự án dài hạn cần bảo trì tốt và dễ dàng cập nhật dữ liệu trong suốt vòng đời phát triển.

\subsection{Realm}
Realm là một cơ sở dữ liệu di động mã nguồn mở được thiết kế để thay thế Core Data với hiệu suất cao và API đơn giản, giúp các nhà phát triển dễ dàng xây dựng ứng dụng với dữ liệu lưu trữ hiệu quả. Đặc biệt, Realm cung cấp một số tính năng nổi bật, làm cho nó trở thành một lựa chọn hấp dẫn trong việc quản lý dữ liệu cho các ứng dụng di động.

\textbf{Đặc điểm chính} của Realm bao gồm khả năng hoạt động đa nền tảng, hỗ trợ cả iOS và Android, giúp các nhà phát triển xây dựng ứng dụng cross-platform dễ dàng hơn. Một tính năng quan trọng khác của Realm là kiến trúc "zero-copy", cho phép truy cập dữ liệu trực tiếp mà không cần sao chép bộ nhớ, mang lại hiệu suất cao. Realm cũng hỗ trợ lập trình reactive, giúp đơn giản hóa việc xử lý các thay đổi dữ liệu và cập nhật UI trong ứng dụng. Một điểm mạnh khác của Realm là tính năng thread-safe theo từng instance của Realm, đảm bảo dữ liệu an toàn khi truy cập từ nhiều luồng khác nhau. Nó còn hỗ trợ mã hóa và đồng bộ hóa dữ liệu thời gian thực, mang lại sự linh hoạt cho các ứng dụng yêu cầu tính năng đồng bộ cao.

\textbf{Ưu điểm} của Realm là API đơn giản và dễ học, giúp tiết kiệm thời gian cho các nhà phát triển khi tích hợp vào ứng dụng. Ngoài ra, hiệu suất của Realm rất cao, điều này rất quan trọng đối với các ứng dụng cần xử lý lượng dữ liệu lớn. Realm còn hỗ trợ đồng bộ hóa dữ liệu theo thời gian thực, một tính năng cần thiết trong các ứng dụng cần cập nhật dữ liệu liên tục. Tài liệu của Realm rất rõ ràng và dễ tiếp cận, cùng với cộng đồng lớn và nhiệt tình hỗ trợ, giúp nhà phát triển giải quyết các vấn đề nhanh chóng.
Mặc dù có nhiều ưu điểm, \textbf{Realm} cũng có \textbf{nhược điểm}. Đầu tiên, Realm không tích hợp sâu với iOS, điều này có thể gây khó khăn khi tích hợp với các framework khác của Apple. Thứ hai, Realm không hỗ trợ struct, thay vào đó yêu cầu sử dụng class kế thừa từ \texttt{Object}, điều này có thể gây ra một số bất tiện cho các nhà phát triển. Thêm vào đó, một số giới hạn trong model threading và truy vấn phức tạp có thể ảnh hưởng đến tính linh hoạt của ứng dụng khi xử lý các tình huống dữ liệu phức tạp.

Về \textbf{khi nên dùng}, Realm là sự lựa chọn tuyệt vời cho các ứng dụng cross-platform, giúp dễ dàng triển khai trên cả iOS và Android mà không gặp phải vấn đề tương thích. Nó cũng rất thích hợp cho các dự án yêu cầu hiệu suất cao và đồng bộ hóa dữ liệu theo thời gian thực, ví dụ như các ứng dụng cần tính năng đồng bộ dữ liệu giữa các thiết bị. Cuối cùng, Realm rất phù hợp cho các dự án ưu tiên offline-first, khi ứng dụng cần có khả năng hoạt động tốt mà không phụ thuộc quá nhiều vào kết nối mạng.

\subsection{SQLite}
SQLite là một hệ quản trị cơ sở dữ liệu quan hệ nhỏ gọn, không yêu cầu server và được sử dụng rộng rãi trên thế giới nhờ vào sự đơn giản, hiệu quả và dễ triển khai. Với thiết kế "self-contained", SQLite không cần cài đặt riêng biệt hoặc máy chủ, khiến nó trở thành một lựa chọn phổ biến cho các ứng dụng di động và ứng dụng nhúng.

\textbf{Đặc điểm chính} của SQLite là tính tự chứa và không yêu cầu cấu hình, giúp dễ dàng tích hợp vào các ứng dụng mà không gặp phải những phức tạp của việc cài đặt và cấu hình server. SQLite rất nhẹ và đáng tin cậy, đồng thời hỗ trợ đa nền tảng (cross-platform), có thể chạy trên nhiều hệ điều hành khác nhau. SQLite hỗ trợ đầy đủ SQL, cho phép thực hiện các thao tác truy vấn cơ sở dữ liệu mạnh mẽ và linh hoạt. Hệ thống này tuân thủ chuẩn ACID, đảm bảo các thao tác dữ liệu được thực hiện một cách an toàn và chính xác, ngay cả khi hệ thống gặp sự cố.

\textbf{Ưu điểm} của SQLite là khả năng kiểm soát toàn diện schema và truy vấn. Điều này có nghĩa là bạn có thể hoàn toàn tùy chỉnh cấu trúc dữ liệu và các câu truy vấn theo nhu cầu của ứng dụng. SQLite cũng rất phù hợp với các truy vấn phức tạp nhờ vào tính mạnh mẽ của SQL. Ngoài ra, SQLite rất dễ dàng để sao lưu và khôi phục, một yếu tố quan trọng trong việc bảo vệ dữ liệu của ứng dụng.

Tuy nhiên, \textbf{SQLite} cũng có \textbf{nhược điểm}. Một trong những khó khăn là bạn phải tự xử lý các vấn đề về thread-safe và migration, vì SQLite không cung cấp các công cụ tự động hóa cho những vấn đề này. SQLite cũng không hỗ trợ object mapping tự động, điều này có thể khiến việc làm việc với các đối tượng trong ứng dụng trở nên phức tạp hơn. Cuối cùng, SQLite không phải là công cụ lý tưởng cho các ứng dụng sử dụng mô hình lập trình reactive, vì nó không hỗ trợ cập nhật dữ liệu theo thời gian thực một cách tự động.

Về \textbf{khi nên dùng}, SQLite là lựa chọn lý tưởng nếu bạn cần kiểm soát hoàn toàn cơ sở dữ liệu và muốn thực hiện các thao tác SQL phức tạp. Nó cũng phù hợp với các ứng dụng không yêu cầu tích hợp chặt với hệ sinh thái iOS, ví dụ như các ứng dụng nhỏ hoặc nhúng, nơi cần sự đơn giản và hiệu suất cao.

\subsection{UserDefaults}
\textbf{Định nghĩa:} \texttt{UserDefaults} là một hệ thống lưu trữ key-value đơn giản, được thiết kế để lưu trữ dữ liệu nhỏ và không nhạy cảm. Đây là công cụ lý tưởng cho việc lưu trữ các cài đặt hoặc trạng thái người dùng trong ứng dụng iOS.

\textbf{Đặc điểm chính:} \texttt{UserDefaults} hỗ trợ lưu trữ dữ liệu dưới dạng key-value. Với API đơn giản, nó cho phép lưu trữ các kiểu cơ bản như String, Int, Bool, Date, và các đối tượng đơn giản khác. Hệ thống này tự động đồng bộ hóa và có cache nội bộ, đồng thời có khả năng đồng bộ với iCloud, cho phép dữ liệu người dùng được lưu trữ và truy cập trên nhiều thiết bị Apple.

\textbf{Ưu điểm:} Với \texttt{UserDefaults}, việc lưu trữ dữ liệu trở nên dễ dàng và hiệu quả. Nó có hiệu suất cao khi lưu trữ các dữ liệu nhỏ và không yêu cầu cấu hình phức tạp. \texttt{UserDefaults} rất thích hợp để lưu trữ trạng thái người dùng hoặc các cài đặt của ứng dụng.

\textbf{Nhược điểm:} Mặc dù dễ sử dụng, \texttt{UserDefaults} không phù hợp cho việc lưu trữ dữ liệu lớn hoặc nhạy cảm vì thiếu các tính năng bảo mật. Thêm vào đó, nó không hỗ trợ các thao tác phức tạp như lọc, truy vấn, hoặc giao dịch dữ liệu, điều này có thể hạn chế trong các ứng dụng cần thao tác với dữ liệu phức tạp.

\textbf{Khi nên dùng:} \texttt{UserDefaults} là lựa chọn lý tưởng khi bạn cần lưu trữ các thông tin đơn giản như theme, ngôn ngữ, âm thanh, hoặc đánh dấu các trạng thái hoàn thành, như đã hoàn thành hướng dẫn sử dụng. Nó cũng rất phù hợp cho việc lưu các cấu hình đơn giản hoặc flags trong ứng dụng.

\subsection{Keychain}
\textbf{Định nghĩa:} \texttt{Keychain} là hệ thống lưu trữ an toàn được thiết kế để bảo vệ thông tin nhạy cảm như mật khẩu, token xác thực và các dữ liệu quan trọng khác. Hệ thống này đảm bảo rằng dữ liệu được mã hóa và chỉ có thể truy cập thông qua các phương thức bảo mật.

\textbf{Đặc điểm chính:} \texttt{Keychain} lưu trữ dữ liệu với mã hóa mạnh mẽ, giúp bảo vệ thông tin nhạy cảm khỏi các mối đe dọa bên ngoài. Dữ liệu trong \texttt{Keychain} tồn tại lâu dài, ngay cả khi ứng dụng bị xóa. Nó hỗ trợ xác thực sinh trắc học như Face ID và Touch ID, đồng thời có thể chia sẻ giữa các ứng dụng của cùng một nhà phát triển. \texttt{Keychain} cũng hỗ trợ đồng bộ hóa với iCloud, cho phép người dùng truy cập dữ liệu bảo mật trên nhiều thiết bị Apple.

\textbf{Ưu điểm:} Một trong những ưu điểm chính của \texttt{Keychain} là mức độ bảo mật cao, làm cho nó trở thành lựa chọn lý tưởng cho việc lưu trữ các thông tin nhạy cảm. Hệ thống này hỗ trợ xác thực bằng Face ID và Touch ID, cung cấp một lớp bảo vệ thêm cho dữ liệu. Hơn nữa, dữ liệu trong \texttt{Keychain} tồn tại lâu dài và có thể được sử dụng trên nhiều thiết bị.

\textbf{Nhược điểm:} Tuy nhiên, \texttt{Keychain} cũng có một số hạn chế. API của nó phức tạp và khó sử dụng hơn so với các giải pháp lưu trữ thông thường, đòi hỏi người phát triển phải có kiến thức chuyên sâu. Thêm vào đó, việc truy cập và thao tác dữ liệu trong \texttt{Keychain} thường chậm hơn so với lưu trữ thông thường. Cuối cùng, việc debug và kiểm tra dữ liệu trong \texttt{Keychain} cũng gặp khó khăn do tính bảo mật cao.

\textbf{Khi nên dùng:} \texttt{Keychain} là giải pháp lý tưởng khi bạn cần lưu trữ các thông tin nhạy cảm như mật khẩu, token xác thực, hoặc các thông tin bảo mật khác. Nó rất phù hợp cho những ứng dụng có yêu cầu bảo mật cao, nơi mà việc đảm bảo an toàn dữ liệu người dùng là ưu tiên hàng đầu.