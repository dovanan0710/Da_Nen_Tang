\section{Kết luận}

Kiến trúc ứng dụng iOS hiện đại không chỉ tập trung vào việc phân tách rõ ràng giữa các tầng chức năng như UI, Business Logic và Data, mà còn chú trọng đến khả năng mở rộng, dễ bảo trì và hiệu suất của ứng dụng. Qua việc áp dụng các mô hình kiến trúc như MVC, MVVM, VIPER, kết hợp cùng các kỹ thuật xử lý bất đồng bộ như \texttt{Completion Handlers}, \texttt{Promises}, \texttt{Async/Await}, và \texttt{Combine}, các nhà phát triển có thể xây dựng những ứng dụng mượt mà, linh hoạt và thân thiện với người dùng.

Từng cách tiếp cận đều có ưu và nhược điểm riêng, đòi hỏi nhà phát triển cần lựa chọn phù hợp theo yêu cầu của dự án và phiên bản iOS hỗ trợ. Trong đó:
\begin{itemize}
    \item \textbf{Completion Handlers} vẫn là lựa chọn đơn giản và phổ biến.
    \item \textbf{Promises} mang lại cú pháp rõ ràng hơn cho các chuỗi bất đồng bộ.
    \item \textbf{Async/Await} đang dần trở thành tiêu chuẩn mới nhờ vào sự rõ ràng và tích hợp sâu trong ngôn ngữ Swift.
    \item \textbf{Combine} mở ra hướng lập trình phản ứng hiện đại, thích hợp cho các ứng dụng nhiều tương tác và dữ liệu động.
\end{itemize}

Việc lựa chọn kiến trúc và công nghệ phù hợp không chỉ giúp tối ưu hóa hiệu suất mà còn tạo ra trải nghiệm người dùng vượt trội. Trong bối cảnh hệ sinh thái Apple không ngừng phát triển, việc nắm vững và ứng dụng linh hoạt các kỹ thuật hiện đại sẽ là chìa khóa giúp các nhà phát triển iOS tạo ra những sản phẩm thành công và bền vững.
