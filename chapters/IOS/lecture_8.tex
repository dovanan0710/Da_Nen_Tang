\section{Kết luận}

\hspace*{0.8cm}Trong kiến trúc ứng dụng iOS hiện đại, không chỉ có sự phân tách rõ ràng giữa các tầng chức năng như UI, Business Logic và Data, mà còn chú trọng đến khả năng mở rộng, dễ bảo trì và hiệu suất của ứng dụng. Mục tiêu này đạt được thông qua việc áp dụng các mô hình kiến trúc như MVC, MVVM, VIPER, kết hợp với các kỹ thuật xử lý bất đồng bộ hiện đại như \texttt{Completion Handlers}, \texttt{Promises}, \texttt{Async/Await}, và \texttt{Combine}. Những kỹ thuật này giúp xây dựng ứng dụng mượt mà, linh hoạt và thân thiện với người dùng.

Mỗi cách tiếp cận này đều có những ưu và nhược điểm riêng, vì vậy, việc lựa chọn phương pháp phù hợp cần dựa trên yêu cầu của dự án và phiên bản iOS mà ứng dụng hỗ trợ. Cụ thể:

\begin{itemize}
\item \textbf{Completion Handlers} vẫn là lựa chọn đơn giản và phổ biến, nhờ vào tính linh hoạt và dễ hiểu.
\item \textbf{Promises} mang lại cú pháp rõ ràng hơn cho các chuỗi bất đồng bộ, giúp mã nguồn dễ theo dõi và bảo trì.
\item \textbf{Async/Await} đang dần trở thành tiêu chuẩn mới nhờ vào sự rõ ràng và tích hợp sâu trong ngôn ngữ Swift, đồng thời cải thiện khả năng đọc hiểu mã nguồn.
\item \textbf{Combine} mở ra hướng lập trình phản ứng hiện đại, đặc biệt phù hợp với các ứng dụng yêu cầu nhiều tương tác và dữ liệu động.
\end{itemize}

Việc lựa chọn kiến trúc và công nghệ phù hợp không chỉ giúp tối ưu hóa hiệu suất mà còn tạo ra trải nghiệm người dùng vượt trội. Trong bối cảnh hệ sinh thái Apple không ngừng phát triển, việc nắm vững và ứng dụng linh hoạt các kỹ thuật hiện đại sẽ là chìa khóa giúp các nhà phát triển iOS tạo ra những sản phẩm thành công và bền vững.