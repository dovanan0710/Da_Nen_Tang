\section{Phân tích chi tiết các kiến trúc phần mềm}

% 
\subsection{Kiến trúc phần mềm ba tầng (Three-tier architecture)}
\renewcommand{\labelitemi}{--}    
    \begin{flushleft}
        \hspace*{0.8cm}Đây là một mô hình tổ chức phần mềm phổ biến, đặc biệt phù hợp với các hệ thống lớn, có khả năng mở rộng và bảo trì lâu dài. Mô hình này phân chia rõ ràng trách nhiệm của từng tầng, từ việc hiển thị, xử lý logic cho đến lưu trữ dữ liệu. Việc áp dụng kiến trúc ba tầng giúp ứng dụng dễ bảo trì, linh hoạt trong mở rộng và tăng khả năng tái sử dụng mã nguồn.
    \end{flushleft}

    \begin{flushleft}
      \hspace*{0.8cm}Kiến trúc ba tầng có thể áp dụng cho nhiều loại ứng dụng khác nhau: từ ứng dụng đơn giản đến phức tạp, từ ứng dụng độc lập đến ứng dụng kết nối mạng. Việc tách biệt ba tầng không chỉ làm cho mã nguồn trở nên rõ ràng hơn mà còn cho phép các nhóm phát triển làm việc độc lập trên từng tầng.
    \end{flushleft}

    \begin{flushleft}
      \hspace*{0.8cm}Tầng trình diễn (Presentation Layer), đây là tầng giao tiếp với người dùng. Chức năng chính của tầng này bao gồm:
      \setlength{\leftmargini}{1.5cm}
      \begin{itemize}
          \item Hiển thị dữ liệu từ tầng nghiệp vụ theo giao diện trực quan.
          \item Nhận lệnh từ người dùng (qua các nút bấm, form, tương tác giao diện).
          \item Không xử lý logic nghiệp vụ, nhờ vậy giao diện có thể dễ dàng tái sử dụng, thay đổi hoặc cập nhật mà không ảnh hưởng đến các tầng khác.
          \item[]$\Rightarrow$ Một lợi ích lớn là khả năng “lắp ghép” lại với các tầng nghiệp vụ khác nhau – giúp cùng một giao diện có thể sử dụng cho nhiều phiên bản khác nhau của hệ thống.
      \end{itemize}
    \end{flushleft}

    \begin{flushleft}
      \hspace*{0.8cm}Tầng nghiệp vụ (Business Logic Layer), tầng này giữ vai trò trung tâm trong hệ thống. Nó thực hiện:
      \setlength{\leftmargini}{1.5cm}
      \begin{itemize}
          \item Chuẩn bị dữ liệu đầu vào để gửi đến tầng dữ liệu.
          \item Chuyển đổi, xử lý dữ liệu nhận về để trả lại cho tầng trình diễn.
          \item Xử lý các lỗi logic hoặc lỗi phản hồi từ tầng dữ liệu.
          \item Áp dụng các quy tắc nghiệp vụ, như kiểm tra hợp lệ, xử lý quy trình.
          \item[]$\Rightarrow$ Tầng này giúp cô lập các xử lý phức tạp khỏi giao diện và dữ liệu, đảm bảo khả năng kiểm thử và bảo trì cao.
      \end{itemize}
    \end{flushleft}

    \begin{flushleft}
      \hspace*{0.8cm}Tầng dữ liệu (Data Layer), là nơi lưu trữ các thông tin quan trọng nhất của ứng dụng:
      \setlength{\leftmargini}{1.5cm}
      \begin{itemize}
          \item Lưu trữ cơ sở dữ liệu (SQL, NoSQL…).
          \item Thực hiện các truy vấn để đảm bảo hiệu năng và độ chính xác cao.
          \item Có thể tích hợp với cơ sở dữ liệu từ xa (server), hệ thống lưu trữ đám mây hoặc tệp cục bộ.
          \item[]$\Rightarrow$ Việc tối ưu tầng dữ liệu giúp cải thiện đáng kể hiệu năng của toàn hệ thống, đặc biệt là trong các ứng dụng có lượng dữ liệu lớn.
      \end{itemize}
    \end{flushleft}

    \begin{flushleft}
      \hspace*{0.8cm}$\Rightarrow$ Kiến trúc ba tầng mang lại lợi ích lớn về mặt tổ chức mã nguồn, bảo trì, kiểm thử và phát triển theo nhóm. Mỗi tầng có trách nhiệm riêng, từ đó giúp ứng dụng dễ mở rộng và thích ứng với thay đổi trong tương lai.
    \end{flushleft}

% 4.2
\subsection{Kiến trúc MVC (Model – View – Controller)}
\renewcommand{\labelitemi}{--}    
    \begin{flushleft}
        \hspace*{0.8cm}MVC (Model – View – Controller) là một mẫu kiến trúc phần mềm cổ điển, phổ biến trong phát triển ứng dụng, đặc biệt là trên nền tảng iOS. Mục tiêu chính của kiến trúc MVC là tách biệt rõ ràng giữa dữ liệu, giao diện và điều khiển xử lý, từ đó giúp ứng dụng dễ bảo trì, mở rộng và nâng cao trải nghiệm người dùng.
    \end{flushleft}

    \begin{flushleft}
      \hspace*{0.8cm}Mô hình này được hình dung như một sơ đồ ba thành phần, mỗi thành phần đảm nhận một vai trò cụ thể:
      \setlength{\leftmargini}{1.5cm}
      \begin{itemize}
          \item Model: Dữ liệu và logic xử lý dữ liệu.
          \item View: Giao diện hiển thị cho người dùng.
          \item Controller: Bộ điều phối, tiếp nhận hành động từ người dùng và điều khiển luồng xử lý.
          \item[]$\Rightarrow$ Ba thành phần hoạt động tách biệt nhưng liên kết chặt chẽ, đảm bảo ứng dụng vận hành trơn tru và dễ dàng điều chỉnh một phần mà không ảnh hưởng đến phần còn lại.
      \end{itemize}
    \end{flushleft}

    \begin{flushleft}
      \hspace*{0.8cm}Model – Mô hình dữ liệu:
      \setlength{\leftmargini}{1.5cm}
      \begin{itemize}
          \item Định danh những gì cần trả về cho người dùng.
          \item Đây là nơi chứa dữ liệu thô, các quy tắc nghiệp vụ và các thao tác xử lý dữ liệu.
          \item Ví dụ: trong một ứng dụng bán hàng, Model chứa thông tin sản phẩm, đơn hàng, người dùng...
      \end{itemize}
    \end{flushleft}

    \begin{flushleft}
      \hspace*{0.8cm}Controller – Bộ điều khiển:
      \setlength{\leftmargini}{1.5cm}
      \begin{itemize}
          \item Tiếp nhận các yêu cầu từ người dùng (qua View).
          \item Thực hiện các truy vấn tài nguyên, gọi các phương thức xử lý trong Model.
          \item Là “bộ não” điều phối mọi hoạt động trong ứng dụng.
      \end{itemize}
    \end{flushleft}

    \begin{flushleft}
      \hspace*{0.8cm}View – Giao diện người dùng:
      \setlength{\leftmargini}{1.5cm}
      \begin{itemize}
          \item Hiển thị dữ liệu dưới dạng dễ hiểu, thân thiện với người dùng.
          \item View không xử lý logic nghiệp vụ, mà chỉ phản hồi lại theo những gì Controller và Model cung cấp.
          \item View sẽ cập nhật nội dung mỗi khi Model thay đổi.
      \end{itemize}
    \end{flushleft}

    \begin{flushleft}
      \hspace*{0.8cm}MVC giúp tách biệt rõ ràng chức năng, dễ dàng phát triển, kiểm thử và bảo trì. Đồng thời cho phép nhiều lập trình viên làm việc song song: người thiết kế giao diện làm View, lập trình viên backend làm Model, còn Controller kết nối hai phần này. Ngoài ra, nó còn có tính tái sử dụng mã nguồn tốt khi một Model có thể được dùng cho nhiều View khác nhau.
    \end{flushleft}

    \begin{flushleft}
      \hspace*{0.8cm}$\Rightarrow$ Mẫu kiến trúc MVC là một giải pháp hiệu quả giúp tổ chức ứng dụng một cách khoa học và linh hoạt. Việc phân chia ứng dụng thành ba thành phần rõ ràng giúp giảm độ phức tạp khi mở rộng, dễ bảo trì, đồng thời nâng cao hiệu quả làm việc nhóm trong quá trình phát triển ứng dụng. Với iOS, MVC vẫn là lựa chọn được ưa chuộng và hỗ trợ tốt trong môi trường phát triển Xcode và Swift.
    \end{flushleft}

% 4.3
\subsection{Kiến trúc MVVM (Model - View - ViewModel)}
\renewcommand{\labelitemi}{--}    
    \begin{flushleft}
        \hspace*{0.8cm}Kiến trúc MVVM (Model – View – ViewModel) là một trong những mẫu thiết kế hiện đại, được áp dụng phổ biến trong phát triển ứng dụng Android (đặc biệt là với sự hỗ trợ từ Jetpack và Kotlin). MVVM ra đời nhằm tối ưu quá trình phát triển ứng dụng bằng cách tách biệt logic hiển thị và logic xử lý, đồng thời tăng tính tự động hóa trong việc cập nhật dữ liệu nhờ cơ chế Data Binding.
    \end{flushleft}

    \begin{flushleft}
      \hspace*{0.8cm}MVVM có cấu trúc gần giống với MVC, nhưng thay vì để Controller điều khiển toàn bộ luồng xử lý, MVVM đưa vào một tầng trung gian là ViewModel – chịu trách nhiệm “kết nối thông minh” giữa dữ liệu (Model) và giao diện (View):
      \setlength{\leftmargini}{1.5cm}
      \begin{itemize}
          \item Tương tự như MVC, View hiển thị dữ liệu, Model chứa dữ liệu và logic xử lý.
          \item Tuy nhiên, Controller được thay thế bằng ViewModel, giúp giảm bớt sự ràng buộc giữa View và Model.
      \end{itemize}
    \end{flushleft}

    \begin{flushleft}
      \hspace*{0.8cm}Model – Dữ liệu và logic xử lý:
      \setlength{\leftmargini}{1.5cm}
      \begin{itemize}
          \item Là nơi lưu trữ các dữ liệu chính của ứng dụng (như thông tin người dùng, sản phẩm...).
          \item Xử lý các nghiệp vụ như tính toán, truy xuất dữ liệu từ cơ sở dữ liệu hoặc API.
          \item Model không trực tiếp liên hệ với View, mà thông qua ViewModel.
      \end{itemize}
    \end{flushleft}

    \begin{flushleft}
      \hspace*{0.8cm}View – Giao diện hiển thị:
      \setlength{\leftmargini}{1.5cm}
      \begin{itemize}
          \item Là phần người dùng tương tác trực tiếp (giao diện ứng dụng).
          \item View trong MVVM không xử lý logic nghiệp vụ mà chỉ phản ánh lại các dữ liệu từ ViewModel.
          \item Nhờ vào Data Binding, View có thể tự động cập nhật khi dữ liệu trong ViewModel thay đổi – giúp giảm mã lặp và tăng hiệu suất phát triển.
      \end{itemize}
    \end{flushleft}

    \begin{flushleft}
      \hspace*{0.8cm}ViewModel – Cầu nối thông minh:
      \setlength{\leftmargini}{1.5cm}
      \begin{itemize}
          \item Chứa các Model và chuẩn bị dữ liệu để hiển thị cho View.
          \item Tạo ra các LiveData hoặc Observable để View có thể theo dõi và tự động cập nhật giao diện khi dữ liệu thay đổi.
          \item Đồng thời, ViewModel cũng xử lý việc truyền dữ liệu từ View sang Model, giúp cập nhật ngược lại khi người dùng nhập liệu hoặc thực hiện thao tác.
      \end{itemize}
    \end{flushleft}

    \begin{flushleft}
      \hspace*{0.8cm}Một điểm mạnh nổi bật của MVVM là cơ chế Data Binding – liên kết dữ liệu hai chiều:
      \setlength{\leftmargini}{1.5cm}
      \begin{itemize}
          \item Khi một đối tượng thuộc nhóm View (như EditText) thay đổi, dữ liệu tương ứng trong ViewModel (hoặc Model) cũng tự động cập nhật.
          \item Ngược lại, khi ViewModel thay đổi giá trị, View cũng cập nhật lại ngay lập tức.
          \item[]$\Rightarrow$ Điều này giúp hạn chế lỗi khi cập nhật giao diện và rút ngắn thời gian phát triển, đặc biệt là trong các ứng dụng có nhiều thao tác tương tác dữ liệu.
      \end{itemize}
    \end{flushleft}

    \begin{flushleft}
      \hspace*{0.8cm}$\Rightarrow$ MVVM là một mô hình kiến trúc mạnh mẽ, phù hợp với các ứng dụng hiện đại cần cập nhật giao diện linh hoạt, liên tục. Với sự hỗ trợ từ Data Binding và LiveData (trong Android), ViewModel giúp đơn giản hóa việc xử lý dữ liệu và đồng bộ giao diện, đồng thời giảm sự phụ thuộc giữa các thành phần, nâng cao khả năng bảo trì và mở rộng về sau. MVVM hiện là lựa chọn ưu tiên trong các dự án Android có quy mô từ vừa đến lớn.
    \end{flushleft}

% 4.4
\subsection{Kiến trúc Client/Server}
\renewcommand{\labelitemi}{--}    
    \begin{flushleft}
        \hspace*{0.8cm}Trong thời đại số, đa số các ứng dụng cần trao đổi dữ liệu qua Internet. Mô hình Client/Server trở thành kiến trúc không thể thiếu, đặc biệt với các ứng dụng có tính năng đồng bộ dữ liệu, chia sẻ thông tin theo thời gian thực, hoặc sử dụng tài nguyên trên máy chủ từ xa. Đồng thời, cần kết hợp với các kiến trúc nội bộ như MVC hoặc MVVM để tối ưu hóa việc xây dựng giao diện và xử lý logic.
    \end{flushleft}

    \begin{flushleft}
      \hspace*{0.8cm}Kiến trúc Client/Server mô tả mô hình trong đó ứng dụng Client (thiết bị người dùng) gửi yêu cầu đến Server (máy chủ từ xa), thường qua HTTP Request, WebSocket, hoặc Web Service. Các chức năng chính bao gồm:
      \setlength{\leftmargini}{1.5cm}
      \begin{itemize}
          \item Client: Gửi yêu cầu (request), hiển thị dữ liệu, tương tác với người dùng.
          \item Server: Xử lý yêu cầu, truy cập cơ sở dữ liệu, trả về dữ liệu kết quả.
          \item[]$\Rightarrow$ Kiến trúc này phù hợp cho các ứng dụng có nhiều người dùng, cần chia sẻ dữ liệu như mạng xã hội, ứng dụng ngân hàng, thương mại điện tử…
      \end{itemize}
    \end{flushleft}