\section{Cơ Sở Lý Thuyết }

% 2.1
\subsection{Khái niệm kiến trúc đa nền tảng}
\renewcommand{\labelitemi}{--}    
\begin{flushleft}
  \subsubsection{Định nghĩa}
    \begin{flushleft}
      \hspace*{0.8cm}Kiến trúc đa nền tảng (cross-platform architecture) là phương pháp phát triển ứng dụng sử dụng một codebase duy nhất để triển khai trên nhiều hệ điều hành (iOS, Android) và nền tảng (web, desktop). Thay vì viết mã riêng biệt cho từng nền tảng, lập trình viên tập trung vào một hệ thống mã nguồn chung, sau đó sử dụng các công cụ hoặc framework để biên dịch và tối ưu hóa cho từng môi trường đích.
    \end{flushleft}

    \begin{flushleft}
      \hspace*{0.8cm}Ví dụ:
      \setlength{\leftmargini}{1.5cm}
      \begin{itemize}
          \item React Native: Mã JavaScript được biên dịch thành native code (Objective-C/Swift cho iOS, Java/Kotlin cho Android) thông qua cơ chế "bridge".
          \item Flutter: Sử dụng ngôn ngữ Dart và Skia Engine để render UI độc lập, không phụ thuộc vào hệ điều hành.
      \end{itemize}
    \end{flushleft}

  \subsubsection{Nguyên tắc "Write Once, Run Anywhere" (WORA)}
    \begin{flushleft}
      \hspace*{0.8cm}Nguyên tắc này được Sun Microsystems giới thiệu từ những năm 1990, nhấn mạnh vào khả năng tái sử dụng mã nguồn tối đa và giảm thiểu công sức phát triển. WORA dựa trên hai yếu tố chính:
      \setlength{\leftmargini}{1.5cm}
      \begin{itemize}
        \item Tính độc lập nền tảng: Code không bị ràng buộc bởi hệ điều hành hoặc phần cứng cụ thể.
        \item Tính nhất quán: Logic nghiệp vụ và giao diện người dùng được thiết kế để hoạt động đồng nhất trên mọi thiết bị.
      \end{itemize}
    \end{flushleft}

    \begin{flushleft}
      \hspace*{0.8cm}Ứng dụng thực tế:
      \setlength{\leftmargini}{1.5cm}
      \begin{itemize}
          \item Microsoft Teams: Sử dụng React Native để triển khai ứng dụng trên iOS, Android và Windows với 90\% code chung.
          \item Google Pay: Flutter cho phép người dùng thanh toán trên cả mobile và web từ một codebase duy nhất.
      \end{itemize}
    \end{flushleft}

  \subsubsection{Lợi ích của kiến trúc đa nền tảng}
    \begin{flushleft}
      \hspace*{0.8cm}Tiết kiệm thời gian và chi phí
      \setlength{\leftmargini}{1.5cm}
      \begin{itemize}
        \item Thời gian phát triển: Giảm 50–80\% so với phương pháp native (theo InfoQ, 2022).
        \item Chi phí nhân lực: Chỉ cần một nhóm phát triển thay vì nhiều nhóm chuyên biệt.
        \item Ví dụ: Startup DeliveryNow tiết kiệm \$300,000 trong 12 tháng nhờ sử dụng Flutter để xây dựng ứng dụng cho cả iOS và Android.
      \end{itemize}
    \end{flushleft}

    \begin{flushleft}
      \hspace*{0.8cm}Dễ dàng bảo trì và mở rộng
      \setlength{\leftmargini}{1.5cm}
      \begin{itemize}
          \item Cập nhật đồng bộ: Sửa lỗi hoặc thêm tính năng mới chỉ cần thực hiện một lần trên codebase chung.
          \item Tích hợp CI/CD: Tự động hóa quy trình triển khai giúp giảm rủi ro và tăng tốc độ phát hành.
      \end{itemize}
    \end{flushleft}

  \subsubsection{Thách thức và hạn chế}
    \begin{flushleft}
      \hspace*{0.8cm}Hiệu năng không bằng native
      \setlength{\leftmargini}{1.5cm}
      \begin{itemize}
        \item Xử lý đồ họa nặng: Ứng dụng đa nền tảng thường chậm hơn 15–30\% so với native khi xử lý animation phức tạp hoặc game 3D (theo Biorn-Hansen, 2021).
        \item Ví dụ: Pokémon GO ban đầu thử nghiệm với Unity (cross-platform) nhưng phải chuyển sang native do lag khi render map 3D.
      \end{itemize}
    \end{flushleft}

    \begin{flushleft}
      \hspace*{0.8cm}Khó tùy chỉnh giao diện
      \setlength{\leftmargini}{1.5cm}
      \begin{itemize}
          \item UI không nhất quán: Các framework đa nền tảng thường sử dụng UI tổng quát, khó đáp ứng đặc thù thiết kế của từng nền tảng (Material Design cho Android, Cupertino cho iOS).
          \item Ví dụ: Spotify (React Native) phải viết lại một số thành phần UI bằng native code để đảm bảo trải nghiệm mượt mà.
      \end{itemize}
    \end{flushleft}

    \begin{flushleft}
      \hspace*{0.8cm}Phụ thuộc vào cộng đồng và công cụ
      \setlength{\leftmargini}{1.5cm}
      \begin{itemize}
          \item Plugin không ổn định: Nhiều thư viện phụ thuộc vào cộng đồng, dễ gặp lỗi hoặc ngừng hỗ trợ.
          \item Ví dụ: Plugin React Native Maps từng gặp lỗi hiển thị trên iOS 14, khiến nhiều ứng dụng bị crash.
      \end{itemize}
    \end{flushleft}

  \subsubsection{Công cụ phát triển}
    \begin{flushleft}
      \hspace*{0.8cm}Phát triển ứng dụng đa nền tảng yêu cầu sử dụng các công cụ chuyên biệt để tối ưu quy trình xây dựng, kiểm thử và triển khai. Dưới đây là các công cụ phổ biến cho từng nền tảng: (Có một cái bảng ở đây)
    \end{flushleft}

    \begin{flushleft}
      \hspace*{0.8cm}Giải thích:
      \setlength{\leftmargini}{1.5cm}
      \begin{itemize}
        \item IDE: Android Studio và XCode là công cụ chính thức cho phát triển native. IntelliJ IDEA hỗ trợ nâng cao cho dự án phức tạp.
        \item Bộ giả lập: Mô phỏng đa dạng thiết bị để kiểm tra ứng dụng trên các điều kiện phần cứng khác nhau.
      \end{itemize}
    \end{flushleft}
\end{flushleft}

% 2.2
\subsection{Các yếu tố quyết định khi lựa chọn đa nền tảng}
\renewcommand{\labelitemi}{--}    
    \begin{flushleft}
      \subsubsection{Phân tích nhóm người dùng mục tiêu}
        \begin{flushleft}
          \hspace*{0.8cm}Thị phần hệ điều hành:
          \setlength{\leftmargini}{1.5cm}
          \begin{itemize}
            \item iOS: Chiếm 27\% thị trường toàn cầu, phổ biến ở Mỹ và Châu Âu.
            \item Android: Chiếm 73\%, thống trị tại Châu Á và Châu Phi (Statista, 2023).
          \end{itemize}
        \end{flushleft}

        \begin{flushleft}
          \hspace*{0.8cm}Chiến lược tiếp cận:
          \setlength{\leftmargini}{1.5cm}
          \begin{itemize}
              \item Nếu nhắm đến người dùng cao cấp (iOS), ưu tiên framework hỗ trợ thiết kế Cupertino (Flutter).
              \item Nếu nhắm đến thị trường đại chúng (Android), React Native phù hợp hơn nhờ tích hợp dễ dàng với Google Services.
          \end{itemize}
        \end{flushleft}

        \begin{flushleft}
          \hspace*{0.8cm}Case study:
          \setlength{\leftmargini}{1.5cm}
          \begin{itemize}
              \item Grab (Flutter): Tập trung vào thị trường Đông Nam Á (đa số dùng Android) nhưng vẫn đảm bảo trải nghiệm mượt mà trên iOS.
          \end{itemize}
        \end{flushleft}

      \subsubsection{Mô hình "Rẻ – Nhanh – Tốt"}
        \begin{flushleft}
          \hspace*{0.8cm}Theo nguyên tắc Iron Triangle trong quản lý dự án, chỉ có thể đạt 2/3 tiêu chí: (Có một cái bảng ở đây)
        \end{flushleft}

        \begin{flushleft}
          \hspace*{0.8cm}Giải thích
          \setlength{\leftmargini}{1.5cm}
          \begin{itemize}
              \item React Native: Phù hợp cho dự án cần MVP (Minimum Viable Product) nhanh chóng, nhưng hiệu năng không cao.
              \item Flutter: Đòi hỏi đầu tư ban đầu để học Dart, nhưng đổi lại hiệu năng và UI tốt hơn.
          \end{itemize}
        \end{flushleft}

        \subsubsection{Khả năng tích hợp với hệ sinh thái hiện có}
        \begin{flushleft}
          \hspace*{0.8cm}React Native: Tận dụng hệ sinh thái JavaScript (Node.js, npm, Expo) và dễ tích hợp với ứng dụng web.
        \end{flushleft}

        \begin{flushleft}
          \hspace*{0.8cm}Flutter: Độc lập hơn nhưng có thể kết hợp với Firebase, Google Cloud qua plugin.
        \end{flushleft}

        \begin{flushleft}
          \hspace*{0.8cm}Ví dụ:
          \setlength{\leftmargini}{1.5cm}
          \begin{itemize}
              \item Shopify sử dụng React Native để tích hợp ứng dụng mobile với nền tảng web sẵn có.
          \end{itemize}
        \end{flushleft}
    \end{flushleft}

% 2.3
\subsection{Lịch sử phát triển của kiến trúc đa nền tảng}
\renewcommand{\labelitemi}{--}    
    \begin{flushleft}
      \subsubsection{Thế hệ đầu tiên (2010–2015): WebView-based Frameworks}
        \begin{flushleft}
          \hspace*{0.8cm}Công cụ tiêu biểu: PhoneGap, Cordova, Ionic.
        \end{flushleft}

        \begin{flushleft}
          \hspace*{0.8cm}Cơ chế hoạt động: Đóng gói ứng dụng dưới dạng WebView để hiển thị nội dung HTML/CSS/JavaScript.
        \end{flushleft}

        \begin{flushleft}
          \hspace*{0.8cm}Ưu điểm:
          \setlength{\leftmargini}{1.5cm}
          \begin{itemize}
              \item Dễ học cho lập trình viên web.
              \item Chi phí thấp.
          \end{itemize}
        \end{flushleft}

        \begin{flushleft}
          \hspace*{0.8cm}Hạn chế:
          \setlength{\leftmargini}{1.5cm}
          \begin{itemize}
              \item Hiệu năng thấp: Không xử lý được animation phức tạp.
              \item Giao diện kém: UI giống trang web, không tương tác được với native features (camera, GPS).
          \end{itemize}
        \end{flushleft}

        \begin{flushleft}
          \hspace*{0.8cm}Ví dụ thất bại:
          \setlength{\leftmargini}{1.5cm}
          \begin{itemize}
              \item Ứng dụng Uber ban đầu dùng Cordova nhưng phải chuyển sang native do lag khi hiển thị bản đồ.
          \end{itemize}
        \end{flushleft}
      
      \subsubsection{Thế hệ thứ hai (2015–2017): Hybrid Frameworks}
        \begin{flushleft}
          \hspace*{0.8cm}Công cụ tiêu biểu: Xamarin, NativeScript.
        \end{flushleft}

        \begin{flushleft}
          \hspace*{0.8cm}Cơ chế hoạt động: Kết hợp WebView với native components thông qua bridge.
        \end{flushleft}

        \begin{flushleft}
          \hspace*{0.8cm}Ưu điểm:
          \setlength{\leftmargini}{1.5cm}
          \begin{itemize}
              \item Cải thiện hiệu năng so với thế hệ đầu.
              \item Truy cập được một số native API.
          \end{itemize}
        \end{flushleft}

        \begin{flushleft}
          \hspace*{0.8cm}Hạn chế:
          \setlength{\leftmargini}{1.5cm}
          \begin{itemize}
              \item Phức tạp trong cấu hình.
              \item Vẫn phụ thuộc vào WebView cho một số tác vụ.
          \end{itemize}
        \end{flushleft}

        \begin{flushleft}
          \hspace*{0.8cm}Ví dụ:
          \setlength{\leftmargini}{1.5cm}
          \begin{itemize}
              \item Microsoft Outlook sử dụng Xamarin để phát triển ứng dụng đa nền tảng.
          \end{itemize}
        \end{flushleft}

      \subsubsection{Thế hệ hiện đại (2017–nay): Native-Reactive Frameworks}
        \begin{flushleft}
          \hspace*{0.8cm}Thế hệ hiện đại (2017–nay): Native-Reactive Frameworks
        \end{flushleft}

        \begin{flushleft}
          \hspace*{0.8cm}Cơ chế hoạt động:
          \setlength{\leftmargini}{1.5cm}
          \begin{itemize}
              \item React Native: Sử dụng JavaScript và Native Modules để render UI qua native components.
              \item Flutter: Sử dụng Dart và Skia Engine để render UI độc lập, không phụ thuộc vào hệ điều hành.
          \end{itemize}
        \end{flushleft}

        \begin{flushleft}
          \hspace*{0.8cm}Ưu điểm:
          \setlength{\leftmargini}{1.5cm}
          \begin{itemize}
              \item Hiệu năng gần native.
              \item Hỗ trợ đa nền tảng (mobile, web, desktop).
          \end{itemize}
        \end{flushleft}

        \begin{flushleft}
          \hspace*{0.8cm}Bước đột phá:
          \setlength{\leftmargini}{1.5cm}
          \begin{itemize}
              \item Flutter 2.0 (2020): Hỗ trợ web và desktop, trở thành framework "đa nền tảng toàn diện".
              \item React Native New Architecture (2022): TurboModules và Fabric giúp tăng tốc độ render.
          \end{itemize}
        \end{flushleft}

        \begin{flushleft}
          \hspace*{0.8cm}Ví dụ thành công:
          \setlength{\leftmargini}{1.5cm}
          \begin{itemize}
              \item Alibaba sử dụng Flutter để xây dựng ứng dụng Xianyu với 200 triệu người dùng, đạt hiệu năng tương đương native.
          \end{itemize}
        \end{flushleft}
    \end{flushleft}

% 2.4
\subsection{Xu hướng tương lai của kiến trúc đa nền tảng}
\renewcommand{\labelitemi}{--}    
  \subsubsection{Tích hợp AI/ML trong phát triển}
    \begin{flushleft}
      \hspace*{0.8cm}Tự động hóa code: Công cụ như Google’s ML Kit cho phép tích hợp machine learning vào ứng dụng đa nền tảng để nhận diện hình ảnh, xử lý ngôn ngữ tự nhiên.
    \end{flushleft}

    \begin{flushleft}
      \hspace*{0.8cm}Tối ưu hiệu năng: AI phân tích code để đề xuất cải thiện FPS hoặc giảm tiêu thụ RAM.
    \end{flushleft}

    \begin{flushleft}
      \hspace*{0.8cm}Ví dụ:
      \setlength{\leftmargini}{1.5cm}
      \begin{itemize}
        \item Adobe XD sử dụng AI để tự động điều chỉnh UI/UX dựa trên hành vi người dùng.
      \end{itemize}
    \end{flushleft}

  \subsubsection{WebAssembly (Wasm) và Progressive Web Apps (PWA)}
    \begin{flushleft}
      \hspace*{0.8cm}WebAssembly: Cho phép ứng dụng chạy trên trình duyệt với tốc độ gần native, mở rộng khả năng đa nền tảng.
    \end{flushleft}

    \begin{flushleft}
      \hspace*{0.8cm}PWA: Kết hợp giữa web và mobile app, hỗ trợ offline và push notifications.
    \end{flushleft}

    \begin{flushleft}
      \hspace*{0.8cm}Ví dụ:
      \setlength{\leftmargini}{1.5cm}
      \begin{itemize}
        \item Starbucks xây dựng PWA để tăng tốc độ tải trang và trải nghiệm người dùng.
      \end{itemize}
    \end{flushleft}

  \subsubsection{Low-Code/No-Code Platforms}
    \begin{flushleft}
      \hspace*{0.8cm}Nền tảng kéo thả: Cho phép người dùng không chuyên tạo ứng dụng đa nền tảng mà không cần viết code.
    \end{flushleft}

    \begin{flushleft}
      \hspace*{0.8cm}Ưu điểm: Giảm thời gian phát triển và chi phí đào tạo.
    \end{flushleft}
    
    \begin{flushleft}
      \hspace*{0.8cm}Ví dụ:
      \setlength{\leftmargini}{1.5cm}
      \begin{itemize}
        \item Microsoft Power Apps giúp doanh nghiệp xây dựng ứng dụng nội bộ chỉ trong vài giờ.
      \end{itemize}
    \end{flushleft}

% 2.5
\subsection{Case Study Chi Tiết}
\renewcommand{\labelitemi}{--}    
    \begin{flushleft}
      \subsubsection{Airbnb và bài học từ React Native}
        \begin{flushleft}
          \hspace*{0.8cm}Thách thức:
          \setlength{\leftmargini}{1.5cm}
          \begin{itemize}
            \item Gặp khó khăn khi tùy chỉnh UI phức tạp cho các nền tảng.
            \item Hiệu năng không đáp ứng được yêu cầu khi mở rộng tính năng đặt phòng thời gian thực.
          \end{itemize}
        \end{flushleft}

        \begin{flushleft}
          \hspace*{0.8cm}Giải pháp: Chuyển sang phát triển native cho các module quan trọng, giữ React Native cho phần admin.
        \end{flushleft}

        \begin{flushleft}
          \hspace*{0.8cm}Kết quả: Cải thiện 30\% hiệu năng nhưng tăng 40\% chi phí bảo trì.
        \end{flushleft}

      \subsubsection{Google Pay: Thành công với Flutter}
        \begin{flushleft}
          \hspace*{0.8cm}Chiến lược: Sử dụng Flutter để triển khai đồng bộ trên iOS, Android và web.
        \end{flushleft}

        \begin{flushleft}
          \hspace*{0.8cm}Lợi ích:
          \setlength{\leftmargini}{1.5cm}
          \begin{itemize}
            \item Giảm 50\% thời gian phát triển.
            \item Đạt FPS ổn định ở mức 60 trên mọi thiết bị.
          \end{itemize}
        \end{flushleft}

        \begin{flushleft}
          \hspace*{0.8cm}Thách thức: Đào tạo đội ngũ về Dart và Skia Engine.
        \end{flushleft}
    \end{flushleft}

% 2.6
\subsection{So Sánh Chi Tiết React Native vs. Flutter}
\renewcommand{\labelitemi}{--}    
    \begin{flushleft}
        \hspace*{0.8cm}(Có một cái bảng ở đây)
    \end{flushleft}

% 2.7
\subsection{Kết luận phần Cơ Sở Lý Thuyết}
\renewcommand{\labelitemi}{--}    
    \begin{flushleft}
        \hspace*{0.8cm}Kiến trúc đa nền tảng đã phát triển qua nhiều giai đoạn, từ các giải pháp dựa trên WebView đến các framework hiện đại như React Native và Flutter. Mỗi công cụ có ưu nhược điểm riêng, phù hợp với từng loại dự án. Việc lựa chọn phụ thuộc vào sự cân nhắc giữa chi phí, thời gian, và chất lượng, cùng với định hướng dài hạn của doanh nghiệp. Xu hướng tương lai hứa hẹn sự tích hợp sâu rộng của AI, WebAssembly và low-code platforms, mở ra kỷ nguyên mới cho phát triển ứng dụng linh hoạt và hiệu quả.
    \end{flushleft}