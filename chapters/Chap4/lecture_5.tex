\section{Cân Nhắc Khi Phát Triển}

\begin{flushleft}
  \hspace*{0.8cm}Khi lựa chọn giữa Flutter và React Native để phát triển ứng dụng di động, các yếu tố kỹ thuật, kinh tế và trải nghiệm người dùng cần được phân tích kỹ lưỡng. Dưới đây là chi tiết từng khía cạnh để hỗ trợ quyết định.
\end{flushleft}

% 5.1
\subsection{Yếu tố kỹ thuật}
\renewcommand{\labelitemi}{--}    
    \begin{flushleft}
        \hspace*{0.8cm}Yếu tố kỹ thuật đóng vai trò quan trọng trong việc đảm bảo ứng dụng hoạt động ổn định, dễ mở rộng và tương thích đa nền tảng. Hai framework này có cách tiếp cận khác biệt về UI, hiệu suất và khả năng tích hợp.
    \end{flushleft}
    \subsubsection{Khả năng tùy biến UI}
    \begin{flushleft}
      \hspace*{0.8cm}Flutter: Flutter sử dụng engine rendering riêng (Skia) và widget được xây dựng từ pixel, cho phép nhà phát triển kiểm soát hoàn toàn giao diện người dùng. Điều này giúp tạo ra các UI độc đáo, phá vỡ giới hạn của components mặc định trên iOS/Android. Ví dụ, ứng dụng Alibaba đã tận dụng Flutter để thiết kế giao diện đồng nhất trên cả hai nền tảng mà không cần chỉnh sửa riêng biệt.
      \setlength{\leftmargini}{1.5cm}
      \begin{itemize}
        \item Ưu điểm: Tùy chỉnh mọi chi tiết, từ animation đến layout. Đồng thời không phụ thuộc vào native components, giảm rủi ro về sự không nhất quán giữa các phiên bản OS.
        \item Nhược điểm: Đòi hỏi nhiều thời gian để thiết kế từ đầu.
      \end{itemize}
    \end{flushleft}

    \begin{flushleft}
      \hspace*{0.8cm}React Native: React Native dựa trên native components (như UIView của iOS hoặc View của Android), nên giao diện mặc định sẽ tuân thủ chuẩn thiết kế của từng nền tảng. Tuy nhiên, để tùy biến sâu, nhà phát triển phải sử dụng thư viện bên thứ ba (React Native Elements, NativeBase) hoặc viết mã native (Swift, Kotlin). Ví dụ, ứng dụng Instagram đã kết hợp React Native với mã native để tối ưu hiệu suất.
      \setlength{\leftmargini}{1.5cm}
      \begin{itemize}
          \item Ưu điểm: UI gần với native, phù hợp ứng dụng cần trải nghiệm "nguyên bản". Tận dụng thư viện có sẵn để tiết kiệm thời gian.
          \item Nhược điểm: Khó đạt được sự đồng nhất giữa các nền tảng nếu không chỉnh sửa riêng.
      \end{itemize}
    \end{flushleft}