
\chapter{Kiến trúc hệ thống}
\label{chap:Chap1}
\vspace{-0.8em}
\hspace*{0.8cm}Kiến trúc hệ thống, vốn giữ vai trò cốt lõi trong việc đảm bảo chất lượng của một ứng dụng di động, chính là chủ đề được khai thác đầu tiên trong tài liệu này. Trong bối cảnh phát triển phần mềm hiện đại, kiến trúc hệ thống không chỉ giúp tổ chức mã nguồn một cách rõ ràng mà còn góp phần nâng cao hiệu suất, khả năng mở rộng và bảo trì của ứng dụng. Chính từ vai trò nền tảng đó, chương mở đầu bằng phần tổng quan về kiến trúc hệ thống trong ứng dụng di động, qua đó giới thiệu các khái niệm cơ bản và phạm vi ứng dụng thực tế. Từ cơ sở này, các yếu tố quan trọng như tính mô-đun, khả năng kiểm thử, mức độ tái sử dụng và tính độc lập của các thành phần được đưa vào phân tích nhằm làm rõ ảnh hưởng của kiến trúc đến toàn bộ vòng đời phát triển ứng dụng. Tiếp nối phần lý thuyết, chương trình bày một số mô hình kiến trúc phổ biến như MVC, MVVM và kiến trúc Client–Server – những mô hình đã được kiểm chứng hiệu quả trong môi trường di động đa nền tảng. Việc áp dụng các mô hình này, như chương đã nhấn mạnh, không thể tách rời khỏi các công nghệ và công cụ hỗ trợ. Do đó, phần cuối của chương tập trung vào việc giới thiệu một số công nghệ hiện hành và công cụ phát triển đang được ưa chuộng, qua đó cung cấp nền tảng kỹ thuật thiết yếu để triển khai thành công một kiến trúc hệ thống hiệu quả và linh hoạt.


\section{Tổng quan về kiến trúc hệ thống}

    \hspace*{0.8cm}Kiến trúc hệ thống là nền tảng quan trọng trong phát triển ứng dụng di động, nơi các thành phần như thiết bị, ứng dụng và server phối hợp hoạt động. Trong phần này, người đọc sẽ tìm hiểu về vai trò của kiến trúc trong việc đảm bảo hiệu suất, khả năng mở rộng và tính ổn định của ứng dụng. Các yếu tố như tính đồng nhất, tính trong suốt và tính khả chuyển sẽ được trình bày như những tiêu chí thiết kế quan trọng. Bên cạnh đó, những mô hình kiến trúc phổ biến và công nghệ hỗ trợ như microservices, containerization, cùng các công cụ DevOps sẽ giúp hình dung rõ hơn về cách xây dựng một hệ thống di động hiện đại và hiệu quả.

    % 1.1.
    \subsection{Tổng quan về kiến trúc hệ thống trong ứng dụng di động}
    \renewcommand{\labelitemi}{--}    
    
    \hspace*{0.8cm}Kiến trúc hệ thống trong phát triển ứng dụng di động là nền tảng kỹ thuật bao gồm ba thành phần chính: phần cứng thiết bị, ứng dụng di động và hệ thống server hỗ trợ phía sau. Mục tiêu chính của kiến trúc này là thiết lập một hệ thống có khả năng vận hành đồng nhất, đảm bảo tính trong suốt giữa các thành phần và đạt được mức độ khả chuyển cao.
    
    \vspace{0.5em}
  
    \hspace*{0.8cm}Việc xây dựng kiến trúc hệ thống hợp lý đóng vai trò quan trọng đối với sự thành công của một ứng dụng. Trước hết, kiến trúc hệ thống ảnh hưởng trực tiếp đến hiệu suất hoạt động cũng như khả năng mở rộng trong tương lai. Bên cạnh đó, một kiến trúc ổn định sẽ giúp ứng dụng hoạt động mượt mà và hạn chế lỗi phát sinh trong quá trình sử dụng. Quan trọng hơn, khi các thành phần trong hệ thống được tổ chức rõ ràng, việc phát hiện và xử lý lỗi sẽ trở nên nhanh chóng và chính xác hơn.
    

    % 1.2.
    \subsection{Các yếu tố quan trọng trong kiến trúc hệ thống}
    \renewcommand{\labelitemi}{--}
    
        \hspace*{0.8cm}Một kiến trúc hệ thống hiệu quả cần đảm bảo ba yếu tố cốt lõi: tính đồng nhất, tính trong suốt và tính khả chuyển.
    
        \vspace{0.5em}
    
        \hspace*{0.8cm}Tính đồng nhất là yêu cầu đầu tiên đối với một hệ thống hiện đại. Để đảm bảo sự nhất quán về dữ liệu giữa client và server, các kiến trúc sư phần mềm thường sử dụng các giao thức và tiêu chuẩn truyền thông như RESTful API hoặc GraphQL \cite{restgraphql}. Các chuẩn này không chỉ hỗ trợ chuẩn hóa quá trình giao tiếp giữa các thành phần mà còn giúp quản lý dữ liệu được lưu trữ và đồng bộ hóa một cách chính xác và hiệu quả trong toàn hệ thống.
      
        \vspace{0.5em}
      
        \hspace*{0.8cm}Tính trong suốt đóng vai trò thiết yếu trong việc phát hiện và xử lý lỗi. Một hệ thống được thiết kế tốt cần có khả năng ghi nhận và thông báo lỗi một cách rõ ràng. Để đạt được điều này, các công cụ như Firebase Crashlytics hoặc Sentry thường được tích hợp vào quá trình vận hành nhằm hỗ trợ theo dõi sự cố theo thời gian thực \cite{firebasecrashlytics}. Đồng thời, kiến trúc hệ thống cũng cần hỗ trợ việc kiểm tra và sửa lỗi một cách nhanh chóng, đảm bảo giảm thiểu gián đoạn trong quá trình vận hành.
      
        \vspace{0.5em}
      
        \hspace*{0.8cm}Tính khả chuyển đề cập đến mức độ linh hoạt của hệ thống khi có sự thay đổi về công nghệ hoặc thành phần. Trong kiến trúc hiện đại, việc áp dụng mô hình microservices hoặc modular architecture cho phép chia tách các thành phần độc lập, từ đó giúp việc thay thế hay nâng cấp trở nên dễ dàng mà không gây ảnh hưởng đến toàn hệ thống \cite{microservices}. Ngoài ra, công nghệ containerization như Docker hoặc Kubernetes cũng được sử dụng phổ biến nhằm tăng khả năng triển khai linh hoạt và đồng nhất giữa các môi trường khác nhau.
      

    % 1.3.
    \subsection{Các mô hình kiến trúc phổ biến trong ứng dụng di động}
    \renewcommand{\labelitemi}{--}
    
        \hspace*{0.8cm}Hiện nay, trong phát triển ứng dụng di động, có một số mô hình kiến trúc được sử dụng phổ biến nhằm đảm bảo tính tổ chức và hiệu quả trong vận hành.
    
        \vspace{0.5em}
    
        \hspace*{0.8cm}Mô hình Layers (hay kiến trúc ba lớp) là một cách tiếp cận cơ bản, trong đó ứng dụng được chia thành ba thành phần: lớp giao diện người dùng (UI), lớp xử lý nghiệp vụ (Business Logic), và lớp dữ liệu (Data). Cách chia lớp rõ ràng này giúp tách biệt các chức năng, tạo điều kiện thuận lợi cho việc bảo trì và mở rộng ứng dụng.
      
        \vspace{0.5em}
      
        \hspace*{0.8cm}Một mô hình quen thuộc khác là MVC (Model-View-Controller), trong đó phần dữ liệu và logic được tách biệt rõ ràng với giao diện và phần xử lý sự kiện. Model chịu trách nhiệm quản lý dữ liệu và logic nghiệp vụ, View đảm nhiệm việc hiển thị giao diện, còn Controller xử lý các hành động của người dùng và điều phối tương tác giữa Model và View.
      
        \vspace{0.5em}
      
        \hspace*{0.8cm}So với MVC, mô hình MVVM (Model-View-ViewModel) cải tiến hơn bằng cách đưa vào thành phần ViewModel nhằm xử lý logic hiển thị. Điều này giúp giảm phụ thuộc giữa View và Model, đồng thời hỗ trợ việc kiểm thử và tái sử dụng mã nguồn hiệu quả hơn.
      
        \vspace{0.5em}
      
        \hspace*{0.8cm}Một mô hình hiện đại nữa là MVI (Model-View-Intent). Mô hình này tiếp cận theo hướng xử lý mọi tương tác của người dùng dưới dạng các Intent. Các Intent này sẽ tạo ra một trạng thái mới được truyền ngược về cho View. Cách tiếp cận này giúp đảm bảo sự đồng bộ trạng thái trong toàn bộ ứng dụng.
      
        \vspace{0.5em}
      
        \hspace*{0.8cm}Cuối cùng, mô hình Client-Server là nền tảng của hầu hết các ứng dụng hiện đại. Ở mô hình này, ứng dụng di động đóng vai trò Client sẽ gửi các yêu cầu đến Server để xử lý, và sau đó nhận phản hồi trở lại. Mô hình này phù hợp với các ứng dụng có nhu cầu giao tiếp dữ liệu thường xuyên với máy chủ hoặc các dịch vụ đám mây.
      

    % 1.4.
    \subsection{Công nghệ và công cụ hỗ trợ}
    \renewcommand{\labelitemi}{--}
    
    \hspace*{0.8cm}Trong một hệ thống phần mềm hiện đại, việc lựa chọn đúng công nghệ và công cụ đóng vai trò then chốt để đảm bảo hiệu quả vận hành cũng như khả năng mở rộng lâu dài.

    \vspace{0.5em}
    
    \hspace*{0.8cm}Ở tầng backend – tức là phần hệ thống chạy trên máy chủ (có thể được đặt cố định tại công ty hoặc triển khai trên nền tảng điện toán đám mây) và chịu trách nhiệm xử lý các yêu cầu từ phía người dùng – các framework phổ biến hiện nay bao gồm Node.js với NestJS, Django, và Spring Boot. Trong đó, NestJS được xây dựng trên nền tảng Node.js và cho phép phát triển ứng dụng phía máy chủ bằng TypeScript, giúp tổ chức mã nguồn hiệu quả hơn \cite{backendframeworks}. Django là một framework mạnh mẽ của Python, nổi bật nhờ tính bảo mật và khả năng mở rộng tốt. Spring Boot là lựa chọn đáng tin cậy trên nền tảng Java, thường được áp dụng cho các hệ thống doanh nghiệp quy mô lớn với cấu trúc phức tạp.
    
    \vspace{0.5em}
      

      %https://www.aphelia.co/blogs/best-backend-for-mobile-app-development-guide
      \begin{figure}[H]
        \centering
        \includegraphics[width=0.75\textwidth]{images/backend_for_mobile_app.jpg}
        \caption{Một số công nghệ phát triển backend cho ứng dụng di động \cite{aphelia2023}.}
        \label{fig:fig1}
      \end{figure}
        \vspace{0.5em}
      
        \hspace*{0.8cm}Về cơ sở dữ liệu, có thể sử dụng các hệ quản trị quan hệ như MySQL hoặc PostgreSQL, tùy theo độ phức tạp của dữ liệu và yêu cầu về hiệu năng. MySQL phù hợp cho các ứng dụng cần tính ổn định cao, trong khi PostgreSQL cung cấp nhiều tính năng nâng cao hơn. Đối với các ứng dụng di động yêu cầu đồng bộ dữ liệu theo thời gian thực, Firebase Firestore là một lựa chọn phù hợp nhờ khả năng cập nhật nhanh và hỗ trợ tốt cho môi trường di động.
      
        \vspace{0.5em}
      
        \hspace*{0.8cm}Phía frontend và mobile, các framework như React Native, Flutter, Swift, và Kotlin là những lựa chọn phổ biến. React Native và Flutter hỗ trợ phát triển ứng dụng đa nền tảng, trong khi Swift và Kotlin là lựa chọn tối ưu cho phát triển ứng dụng gốc trên iOS và Android tương ứng.
      
        \vspace{0.5em}
      
        \hspace*{0.8cm}Cuối cùng, để tự động hóa quy trình triển khai và kiểm thử, các công cụ CI/CD như GitHub Actions hoặc Jenkins thường được tích hợp vào chu trình phát triển phần mềm. CI/CD là viết tắt của Continuous Integration và Continuous Deployment/Delivery, nghĩa là tích hợp liên tục và triển khai liên tục. Trong quy trình này, mã nguồn được kiểm tra, xây dựng và đưa lên môi trường thử nghiệm hoặc sản xuất một cách tự động, giúp phát hiện lỗi sớm và rút ngắn thời gian phát hành phần mềm.
      
   

\section{Lịch sử ngành lập trình di động}
\begin{flushleft}
  \hspace*{0.8cm}Ngành lập trình di động không xuất hiện một cách đột ngột, mà hình thành song song với sự phát triển của điện thoại di động qua từng thời kỳ. Từ những thiết bị to lớn, đắt đỏ và đơn chức năng, đến các smartphone hiện đại với khả năng xử lý mạnh mẽ và đa nhiệm, lập trình di động đã dần trở thành một lĩnh vực quan trọng trong ngành công nghệ thông tin. Việc hiểu rõ bối cảnh và các bước ngoặt trong lịch sử phát triển không chỉ giúp ta nhìn nhận đúng vai trò của lập trình di động, mà còn cho thấy những thách thức và cơ hội trong hành trình tạo ra các ứng dụng phục vụ hàng tỷ người dùng trên toàn cầu.
\end{flushleft}
% 2.1
\subsection{Tổng quan về lịch sử ngành lập trình di động}
\renewcommand{\labelitemi}{--}  

\begin{flushleft}
  \hspace*{0.8cm}Lập trình di động là một lĩnh vực đã phát triển song hành với sự tiến hóa của thiết bị di động trong suốt nhiều thập kỷ. Từ những năm 1990, khi các thiết bị cầm tay đầu tiên như Nokia và BlackBerry bắt đầu phổ biến, nhu cầu về các phần mềm cơ bản như lịch, máy tính và trò chơi đơn giản đã thúc đẩy sự ra đời của các nền tảng lập trình di động ban đầu. Những ứng dụng này chủ yếu được viết bằng ngôn ngữ C hoặc Java ME, giới hạn bởi phần cứng yếu và màn hình nhỏ.
\end{flushleft}

\begin{flushleft}
  \hspace*{0.8cm}Giai đoạn chuyển mình rõ rệt nhất của ngành bắt đầu từ năm 2007, khi Apple giới thiệu iPhone và hệ điều hành iOS. Sự xuất hiện của smartphone với màn hình cảm ứng, kết nối internet mạnh mẽ và kho ứng dụng trực tuyến đã tạo ra một bước ngoặt lớn. Ngay sau đó, Google ra mắt Android – một hệ điều hành mã nguồn mở, cho phép các nhà phát triển dễ dàng tạo ứng dụng và triển khai trên nhiều thiết bị khác nhau.
\end{flushleft}

\begin{flushleft}
  \hspace*{0.8cm}Từ năm 2010 đến nay, lập trình di động đã trở thành một trong những lĩnh vực sôi động nhất của ngành công nghệ thông tin. Các công cụ phát triển như Android Studio, Xcode, cùng với các framework đa nền tảng như React Native, Flutter hay Xamarin, đã giúp rút ngắn thời gian phát triển và mở rộng khả năng tiếp cận người dùng. Đồng thời, những xu hướng mới như trí tuệ nhân tạo (AI), thực tế tăng cường (AR), và Internet vạn vật (IoT) cũng được tích hợp ngày càng sâu vào các ứng dụng di động.
\end{flushleft}

\begin{flushleft}
  \hspace*{0.8cm}Có thể thấy, lịch sử ngành lập trình di động không chỉ phản ánh sự tiến bộ của công nghệ phần cứng và phần mềm, mà còn cho thấy vai trò ngày càng lớn của các ứng dụng trong đời sống hàng ngày. Hiểu rõ bối cảnh lịch sử này là cơ sở quan trọng để các nhà phát triển định hướng tương lai, lựa chọn công cụ phù hợp, và xây dựng các giải pháp sáng tạo đáp ứng nhu cầu người dùng hiện đại.
\end{flushleft}

% 2.2
\subsection{Một số thiết kế của điện thoại di động qua từng thời kỳ}
\renewcommand{\labelitemi}{--}    
    \begin{flushleft}
        \hspace*{0.8cm}Vào những năm 1980, những chiếc điện thoại di động đầu tiên ra đời với thiết kế to, nặng và thường phải mang theo như một chiếc túi xách. Các thiết bị này chủ yếu được sử dụng để nghe và gọi, không có màn hình hiển thị hoặc giao diện tương tác như hiện nay. Lúc này, khái niệm về lập trình di động vẫn chưa xuất hiện vì các thiết bị không hỗ trợ cài đặt thêm phần mềm.
    \end{flushleft}

    \begin{flushleft}
      \hspace*{0.8cm}Đến thập niên 1990, cùng với sự phát triển của công nghệ vi mạch và màn hình LCD, điện thoại di động bắt đầu trở nên nhỏ gọn và phổ biến hơn. Các dòng máy như Nokia 3210 hay Motorola StarTAC không chỉ hỗ trợ gọi điện và nhắn tin, mà còn tích hợp một số ứng dụng đơn giản như đồng hồ báo thức, lịch và trò chơi Snake. Đây là thời kỳ lập trình di động bắt đầu hình thành, chủ yếu dưới dạng các phần mềm nhúng hoặc ứng dụng đơn giản viết bằng ngôn ngữ C hoặc Java ME.
  \end{flushleft}

  \begin{flushleft}
    \hspace*{0.8cm}Giai đoạn tiếp theo bắt đầu từ năm 2007, khi Apple ra mắt iPhone – một thiết bị đánh dấu kỷ nguyên mới của điện thoại thông minh. Với màn hình cảm ứng đa điểm, hiệu năng mạnh mẽ và khả năng kết nối internet, iPhone đã thay đổi hoàn toàn cách người dùng tương tác với thiết bị di động. Sự ra đời của App Store vào năm 2008 đã mở ra một thị trường ứng dụng sôi động, thúc đẩy sự phát triển mạnh mẽ của lập trình di động. Gần như song song, Google cũng phát triển Android – hệ điều hành mã nguồn mở, cho phép các nhà sản xuất và lập trình viên tự do phát triển ứng dụng và tùy biến theo nhu cầu.
\end{flushleft}

\begin{flushleft}
  \hspace*{0.8cm}Từ năm 2010 đến nay, thiết bị di động đã đạt được những bước tiến vượt bậc về hiệu năng, độ phân giải màn hình, dung lượng pin và khả năng xử lý đa nhiệm. Đồng thời, các công nghệ mới như trí tuệ nhân tạo (AI), thực tế ảo (VR), thực tế tăng cường (AR), và cảm biến sinh trắc học cũng được tích hợp ngày càng sâu vào thiết bị. Bên cạnh đó, sự xuất hiện của các nền tảng phát triển đa nền tảng như React Native, Flutter, và Xamarin đã tạo điều kiện thuận lợi cho các nhà phát triển viết ứng dụng một lần và triển khai trên cả iOS lẫn Android.
\end{flushleft}

\begin{flushleft}
  \hspace*{0.8cm}Có thể thấy, thiết bị di động không ngừng được cải tiến cả về phần cứng lẫn phần mềm. Từ một công cụ liên lạc đơn giản, chúng đã trở thành trung tâm của mọi hoạt động số hóa trong đời sống hiện đại – từ học tập, làm việc, giải trí cho đến quản lý tài chính cá nhân. Quá trình phát triển qua các giai đoạn này không chỉ mở ra nhiều cơ hội cho lập trình viên, mà còn đặt ra những thách thức mới về hiệu suất, bảo mật và trải nghiệm người dùng.
\end{flushleft}

% 2.3
\subsection{Motorola DynaTAC 8000X – Bước ngoặt đầu tiên của điện thoại di động}
\renewcommand{\labelitemi}{--}    
\begin{flushleft}
    \hspace*{0.8cm}Motorola DynaTAC 8000X, được giới thiệu vào năm 1983, là chiếc điện thoại di động thương mại đầu tiên trên thế giới. Thiết bị này đã đánh dấu một bước ngoặt quan trọng trong ngành viễn thông toàn cầu khi lần đầu tiên mang lại khả năng liên lạc di động thực sự cho người dùng cá nhân. Với hình dáng to lớn và trọng lượng nặng, DynaTAC vẫn nhanh chóng trở thành biểu tượng công nghệ của thời đại, đặt nền móng cho sự hình thành và phát triển sau này của các thiết bị di động thông minh cũng như ngành lập trình ứng dụng di động.
\end{flushleft}

\begin{flushleft}
  \hspace*{0.8cm}DynaTAC 8000X sở hữu thiết kế nổi bật theo phong cách công nghiệp, với kích thước khoảng 33 x 4.4 x 8.9 cm và trọng lượng lên đến 1.1 kg. Người dùng cần sử dụng cả hai tay để thao tác, khiến nó được gọi với biệt danh quen thuộc là “brick phone” – điện thoại cục gạch. Đây là minh chứng rõ rệt cho giai đoạn khởi đầu của công nghệ di động, khi tính cơ động vẫn còn là một khái niệm tương đối.
\end{flushleft}

\begin{flushleft}
  \hspace*{0.8cm}Về mặt kinh tế, thiết bị này là một sản phẩm xa xỉ. Với mức giá lên tới \$3,995 vào năm 1983 (tương đương hơn \$10,000 hiện nay sau điều chỉnh lạm phát), DynaTAC chỉ phù hợp với tầng lớp thượng lưu hoặc những người có nhu cầu đặc biệt. Không chỉ đắt đỏ về giá mua, người dùng còn phải trả thêm chi phí thuê bao hàng tháng và phí gọi tính theo phút – làm cho việc sử dụng thiết bị trở thành một khoản đầu tư lớn về tài chính.
\end{flushleft}

\begin{flushleft}
  \hspace*{0.8cm}Về mặt kỹ thuật, DynaTAC không sở hữu màn hình màu, cảm ứng hay các chức năng nâng cao. Thiết bị chỉ phục vụ chức năng cơ bản là thực hiện và nhận cuộc gọi. Tuy nhiên, chính việc đưa công nghệ viễn thông ra khỏi không gian cố định – như điện thoại bàn – đã mở ra một thời kỳ hoàn toàn mới, trong đó việc giao tiếp trở nên linh hoạt hơn bao giờ hết. Đó là tiền đề quan trọng để những công nghệ di động sau này phát triển mạnh mẽ hơn.
\end{flushleft}

\begin{flushleft}
  \hspace*{0.8cm}Tuy không hỗ trợ cài đặt phần mềm hay ứng dụng, DynaTAC vẫn có vai trò đặc biệt trong lịch sử lập trình di động. Sự xuất hiện của nó đã cho thấy tiềm năng thương mại rất lớn của các thiết bị cầm tay, từ đó thúc đẩy các nhà sản xuất phần cứng và kỹ sư viễn thông đầu tư vào nghiên cứu, phát triển hệ điều hành và nền tảng phần mềm cho thiết bị di động. Nhu cầu mở rộng chức năng thiết bị đã đặt nền móng cho việc phát triển các ứng dụng nhúng đầu tiên và các hệ điều hành cơ bản dành riêng cho di động.
\end{flushleft}

\begin{flushleft}
  \hspace*{0.8cm}Nhìn một cách tổng thể, DynaTAC 8000X không chỉ là thiết bị điện thoại đầu tiên mà còn là biểu tượng khởi đầu của một ngành công nghiệp đầy tiềm năng. Mặc dù giới hạn về công nghệ và chức năng, sự thành công thương mại của nó đã truyền cảm hứng cho hàng loạt phát minh sau này – từ điện thoại thông minh cho tới hệ sinh thái lập trình ứng dụng di động đa dạng và phong phú như hiện nay.
\end{flushleft}

% 2.4
\subsection{Từ “cục gạch” đến trải nghiệm người dùng hiện đại}
\renewcommand{\labelitemi}{--}    
    \begin{flushleft}
        \hspace*{0.8cm}Sau khi điện thoại di động ra đời vào thập niên 1980, công nghệ đã không ngừng tiến hóa để biến những thiết bị vốn to lớn, nặng nề và đắt đỏ thành các sản phẩm nhỏ gọn, tiện lợi và phổ biến trong đời sống hàng ngày. Quá trình thu nhỏ phần cứng đi kèm với việc mở rộng khả năng xử lý và kết nối đã biến điện thoại di động từ một công cụ liên lạc thành một nền tảng công nghệ mạnh mẽ. Trong bối cảnh đó, trải nghiệm người dùng (User Experience – UX) dần trở thành một yếu tố trọng tâm, không chỉ trong thiết kế phần cứng mà còn trong quá trình phát triển phần mềm ứng dụng.
    \end{flushleft}

    \begin{flushleft}
      \hspace*{0.8cm}Sự phát triển của UX trên thiết bị di động gắn liền với sự thay đổi trong hành vi và kỳ vọng của người dùng. Ban đầu, người dùng điện thoại di động chủ yếu mong muốn một thiết bị có thể thực hiện cuộc gọi ổn định và dễ sử dụng. Tuy nhiên, khi điện thoại trở nên phổ biến và chuyển mình thành điện thoại thông minh, người dùng bắt đầu kỳ vọng nhiều hơn: giao diện phải trực quan, tốc độ xử lý nhanh, thao tác mượt mà và trải nghiệm tổng thể phải liền mạch. Chính sự thay đổi về nhận thức và nhu cầu này đã thúc đẩy ngành lập trình di động phải chuyển hướng từ lập trình chức năng đơn thuần sang thiết kế lấy người dùng làm trung tâm.
    \end{flushleft}

    \begin{flushleft}
      \hspace*{0.8cm}Việc điện thoại trở nên dễ tiếp cận hơn là một trong những yếu tố khiến UX trở thành một vấn đề cốt lõi. Nếu như những chiếc điện thoại đầu tiên có giá hàng nghìn đô la và chỉ phục vụ giới doanh nhân, thì ngày nay, người dùng ở mọi độ tuổi và tầng lớp xã hội đều có thể sở hữu ít nhất một thiết bị di động. Khi số lượng người dùng tăng lên, sự đa dạng về nhu cầu, trình độ công nghệ và thị hiếu sử dụng cũng trở nên phong phú hơn. Điều này tạo ra áp lực lớn đối với các nhà phát triển phần mềm, buộc họ phải tạo ra những sản phẩm vừa đơn giản, dễ dùng, vừa đầy đủ chức năng và mang lại cảm giác sử dụng thoải mái.
  \end{flushleft}

  \begin{flushleft}
    \hspace*{0.8cm}Bên cạnh sự thay đổi của người dùng, sự tiến bộ nhanh chóng của phần cứng cũng là một yếu tố không thể bỏ qua. Các thiết bị hiện đại sở hữu màn hình độ phân giải cao, cảm ứng đa điểm, bộ xử lý mạnh, RAM lớn và pin dung lượng cao. Những tiến bộ này mở ra cơ hội cho các nhà phát triển phần mềm thiết kế giao diện ngày càng sinh động, đẹp mắt và tương tác tốt hơn. Nếu như trước đây ứng dụng di động chỉ có thể hiển thị văn bản đơn giản hoặc hình ảnh tĩnh, thì hiện nay, chúng có thể tích hợp video, hiệu ứng chuyển động, phản hồi xúc giác và thậm chí cả trí tuệ nhân tạo để cá nhân hóa trải nghiệm người dùng.
  \end{flushleft}

  \begin{flushleft}
    \hspace*{0.8cm}Sự thay đổi lớn cũng diễn ra trong vai trò của các nhà sản xuất thiết bị di động. Trong giai đoạn đầu, các nhà sản xuất phần cứng thường là những người duy nhất phát triển phần mềm cho thiết bị của họ, từ giao diện người dùng cho đến các chức năng cơ bản. Tuy nhiên, mô hình này nhanh chóng bộc lộ hạn chế khi không thể đáp ứng kịp tốc độ thay đổi của thị trường và nhu cầu người dùng. Đồng thời, các nhà sản xuất cũng không sẵn sàng chia sẻ chi tiết về phần cứng của họ vì lý do bảo mật và cạnh tranh, khiến cho bên thứ ba rất khó tiếp cận và phát triển phần mềm.
  \end{flushleft}

  \begin{flushleft}
    \hspace*{0.8cm}Chính sự bất cập đó đã làm nảy sinh một nhu cầu mới: cần có một lớp trung gian – hay chính xác hơn là một chuẩn kỹ thuật – giúp kết nối phần mềm với phần cứng một cách hiệu quả và an toàn. Các API (Application Programming Interface), SDK (Software Development Kit) và hệ điều hành di động như iOS hay Android đã ra đời nhằm mục đích đó. Chúng đóng vai trò là cầu nối giữa nhà sản xuất thiết bị và cộng đồng lập trình viên, giúp quá trình phát triển phần mềm trở nên dễ dàng hơn mà không cần phải hiểu sâu về kiến trúc phần cứng.
  \end{flushleft}

  \begin{flushleft}
    \hspace*{0.8cm}Một trong những giải pháp mở đầu cho quá trình này là việc sử dụng trình duyệt web di động như một nền tảng để triển khai ứng dụng. Thay vì phát triển phần mềm cài đặt riêng biệt, các lập trình viên bắt đầu tối ưu hóa trang web để hoạt động tốt trên điện thoại. Từ đó, khái niệm “ứng dụng web di động” ra đời, đóng vai trò như bước chuyển tiếp giữa trình duyệt truyền thống và ứng dụng gốc (native app). Những trải nghiệm đầu tiên trên nền tảng web đã đặt nền móng cho việc hình thành tư duy thiết kế giao diện người dùng hiện đại.
  \end{flushleft}

  \begin{flushleft}
    \hspace*{0.8cm}Nhìn chung, sự phát triển của UX trên thiết bị di động không chỉ phản ánh tiến bộ của công nghệ mà còn cho thấy mối liên kết chặt chẽ giữa phần cứng, phần mềm và hành vi người dùng. Việc đặt người dùng làm trung tâm không chỉ là xu hướng tạm thời mà đã trở thành triết lý phát triển cốt lõi trong ngành lập trình di động hiện đại. Từ “cục gạch” thô sơ ngày xưa đến những chiếc điện thoại tinh xảo hiện nay, trải nghiệm người dùng chính là yếu tố quyết định sự thành bại của một sản phẩm trong thời đại số.
  \end{flushleft}

% 2.5
\subsection{Chuẩn WAP – Khởi đầu cho trình duyệt web trên di động}
\renewcommand{\labelitemi}{--}    
\begin{flushleft}
  \hspace*{0.8cm}Trong những năm cuối thập niên 1990 và đầu những năm 2000, khi điện thoại di động bắt đầu được sử dụng rộng rãi trên toàn cầu, nhu cầu truy cập thông tin mọi lúc mọi nơi cũng trở nên cấp thiết. Tuy nhiên, hạ tầng mạng di động ở thời kỳ này lại chưa đủ phát triển: tốc độ kết nối rất chậm, băng thông hạn chế, và độ ổn định thấp. Vì vậy, việc sử dụng các trang web HTML tiêu chuẩn như trên máy tính là điều gần như không khả thi trên thiết bị di động lúc bấy giờ.
  \end{flushleft}
  
  \begin{flushleft}
  \hspace*{0.8cm}Để giải quyết những hạn chế đó, chuẩn WAP – Wireless Application Protocol đã được ra đời như một giải pháp trung gian nhằm hiện thực hóa giấc mơ duyệt web trên thiết bị di động. WAP là một giao thức truyền thông được thiết kế đặc biệt cho môi trường mạng không dây với mục tiêu tạo điều kiện cho các thiết bị di động có thể truy cập, nhận và hiển thị dữ liệu từ các máy chủ web.
  \end{flushleft}
  
  \begin{flushleft}
  \hspace*{0.8cm}Khác với giao thức HTTP vốn được xây dựng cho các máy tính kết nối mạng ổn định và tốc độ cao, WAP được ví như một “phiên bản rút gọn” của HTTP, tối ưu hóa cho việc vận hành trên các thiết bị có tài nguyên phần cứng hạn chế và kết nối mạng chậm như 2G hay GPRS. Nhờ vậy, dù rất đơn giản và sơ khai, WAP đã đóng vai trò như chiếc cầu nối giữa thế giới web truyền thống và thiết bị di động.
  \end{flushleft}
  
  \begin{flushleft}
  \hspace*{0.8cm}Một trong những yếu tố then chốt làm nên hiệu quả của WAP chính là ngôn ngữ đánh dấu WML (Wireless Markup Language). Thay vì sử dụng HTML, vốn quá nặng và phức tạp với các thiết bị di động thời bấy giờ, WML được thiết kế nhẹ hơn nhiều, loại bỏ các yếu tố đồ họa và định dạng phức tạp. Cú pháp của WML khá giống HTML nhưng bị giới hạn về số lượng thẻ và khả năng tương tác. Điều này cho phép hiển thị nội dung văn bản đơn giản như tiêu đề, đoạn văn, và các liên kết mà vẫn đảm bảo tốc độ tải nhanh và khả năng tương thích với phần cứng yếu.
  \end{flushleft}
  
  \begin{flushleft}
  \hspace*{0.8cm}Thêm vào đó, kiến trúc của WAP được xây dựng xoay quanh nguyên tắc tối ưu cho mạng yếu, nhằm giảm thiểu tối đa dữ liệu cần truyền và yêu cầu xử lý phức tạp. Nhờ đó, các thiết bị di động có thể tải và hiển thị nội dung một cách hiệu quả hơn, dù đang hoạt động trong điều kiện mạng không ổn định và băng thông hạn chế.
  \end{flushleft}
  
  \begin{flushleft}
  \hspace*{0.8cm}Trong thời kỳ đầu, nhiều tổ chức và hãng thông tấn đã nhanh chóng nhận ra tiềm năng của WAP và bắt đầu phát triển các phiên bản website hỗ trợ giao thức này. CNN là một trong những đơn vị tiên phong, cung cấp nội dung tin tức thời sự cho người dùng di động thông qua nền tảng WAP. Tương tự, ESPN cũng phát triển phiên bản WAP nhằm cập nhật tỉ số thể thao và thông tin thi đấu cho khán giả thường xuyên di chuyển.
  \end{flushleft}
  
  \begin{flushleft}
  \hspace*{0.8cm}Những ví dụ kể trên là minh chứng rõ ràng cho việc các nhà phát triển ứng dụng web đã bắt đầu chú trọng đến nền tảng di động, dù với rất nhiều giới hạn kỹ thuật so với web truyền thống trên máy tính. Tuy vậy, chính những nỗ lực này đã mở ra một hướng đi mới cho lĩnh vực phát triển ứng dụng – hướng đi mà sau này sẽ trở thành một trụ cột quan trọng trong ngành công nghệ.
  \end{flushleft}
  
  \begin{flushleft}
  \hspace*{0.8cm}Xét về mặt lịch sử và ý nghĩa đối với lập trình di động, WAP có thể được xem là cột mốc đánh dấu sự khởi đầu của khái niệm "ứng dụng web di động". Nhờ vào chuẩn này, các lập trình viên thời kỳ đầu đã có môi trường và công cụ để bắt đầu phát triển nội dung dành riêng cho điện thoại di động, dù chỉ ở mức rất cơ bản.
  \end{flushleft}
  
  \begin{flushleft}
  \hspace*{0.8cm}Không những thế, việc áp dụng WAP còn giúp định hình tư duy thiết kế ứng dụng tối giản, ưu tiên tốc độ và hiệu năng – một nguyên tắc vẫn còn giá trị trong phát triển ứng dụng di động hiện đại. Đồng thời, nó cũng mở ra kỷ nguyên mà điện thoại không chỉ là thiết bị nghe gọi, mà còn là cổng kết nối đến thế giới thông tin số – nơi người dùng có thể cập nhật tin tức, tra cứu thông tin, hay tương tác với nội dung web ở bất cứ đâu.
  \end{flushleft}

% 2.6
\subsection{Sự phát triển của thanh toán di động và bước ngoặt trong lập trình ứng dụng}
\renewcommand{\labelitemi}{--}

\begin{flushleft}
\hspace*{0.8cm}Khi điện thoại di động dần trở thành một phần thiết yếu trong đời sống hàng ngày, nhu cầu không chỉ dừng lại ở liên lạc mà còn mở rộng sang các tiện ích phục vụ tiêu dùng và giải trí. Trong quá trình phát triển đó, một yếu tố có vai trò đặc biệt quan trọng trong việc thúc đẩy sự bùng nổ của ứng dụng di động chính là khả năng thanh toán ngay trên thiết bị.
\end{flushleft}

\begin{flushleft}
\hspace*{0.8cm}Trong những ngày đầu, hình thức thanh toán thông qua SMS (Short Message Service) được xem là giải pháp khả thi nhất, bởi vì hầu hết các điện thoại di động đều hỗ trợ gửi và nhận tin nhắn. Với mô hình này, người dùng chỉ cần gửi tin nhắn đến một đầu số dịch vụ như 8xxx hoặc 6xxx để đổi lại các nội dung số như nhạc chuông, hình nền, hình động hay truyện cười. Khoản phí của tin nhắn này cao hơn nhiều so với tin nhắn thông thường và được tính trực tiếp vào tài khoản di động.
\end{flushleft}

\begin{flushleft}
\hspace*{0.8cm}Những tin nhắn dạng này được gọi là VAS – Value Added Services (dịch vụ giá trị gia tăng), và chính chúng đã đặt nền móng cho mô hình kinh doanh nội dung số trên nền tảng di động. Đây cũng là lần đầu tiên mà lập trình viên có thể kiếm tiền thông qua các ứng dụng hoặc nội dung do họ xây dựng, dù vẫn còn ở dạng rất đơn giản.
\end{flushleft}

\begin{flushleft}
\hspace*{0.8cm}Khi điện thoại phát triển và người dùng ngày càng quen với việc tiêu dùng nội dung số, nhiều hình thức thanh toán khác cũng nhanh chóng ra đời để thay thế hoặc bổ sung cho SMS, tạo điều kiện mở rộng mô hình kinh doanh ứng dụng. Một trong số đó là thẻ cào điện thoại, cho phép người dùng nạp tiền vào tài khoản chính rồi sử dụng số dư này để mua vật phẩm hoặc dịch vụ trong game và ứng dụng.
\end{flushleft}

\begin{flushleft}
\hspace*{0.8cm}Tiếp theo là web charging, một hình thức thanh toán hiện đại hơn, cho phép người dùng mua nội dung trực tiếp trên website hoặc ứng dụng mà không cần can thiệp thủ công. Khi người dùng chọn một dịch vụ trên trình duyệt hoặc trong app, hệ thống sẽ tự động trừ tiền vào tài khoản di động mà không yêu cầu xác nhận nhiều bước. Dù còn hạn chế và dễ phát sinh lỗi, web charging đã cho thấy tiềm năng to lớn của thương mại điện tử di động trong tương lai.
\end{flushleft}

\begin{flushleft}
\hspace*{0.8cm}Mặc dù các hình thức thanh toán kể trên còn đơn giản và chưa được bảo mật cao, nhưng chính chúng đã mở ra cánh cửa đầu tiên cho khái niệm “thanh toán di động”. Từ đó, lập trình viên không chỉ viết ứng dụng phục vụ giải trí mà còn có cơ hội tạo ra giá trị kinh tế thực sự, khuyến khích việc đầu tư bài bản vào thiết kế, trải nghiệm người dùng và chiến lược kiếm tiền.
\end{flushleft}

\begin{flushleft}
\hspace*{0.8cm}Song song với sự phát triển của các mô hình thanh toán, ngành lập trình di động cũng chứng kiến một bước chuyển mình mạnh mẽ. Khi người dùng tăng trưởng nhanh chóng và nhu cầu sử dụng di động vượt xa kỳ vọng, các công ty công nghệ lớn như Apple, Google, và sau này là Samsung, Huawei đã chính thức bước vào cuộc chơi với các hệ điều hành và nền tảng ứng dụng riêng biệt.
\end{flushleft}

\begin{flushleft}
\hspace*{0.8cm}Trước giai đoạn này, lập trình di động vẫn chủ yếu dựa vào nền tảng J2ME (Java 2 Micro Edition) – một môi trường phát triển cực kỳ rối rắm. Các lập trình viên phải đối mặt với hàng loạt thách thức: mỗi dòng điện thoại lại có độ phân giải khác nhau, hệ thống phím bấm không đồng nhất, dung lượng bộ nhớ thấp và phần cứng yếu. Công cụ hỗ trợ thì thiếu thốn, giao diện đơn điệu và không có khả năng tái sử dụng mã nguồn.
\end{flushleft}

\begin{flushleft}
\hspace*{0.8cm}Chỉ đến khi phần cứng của thiết bị di động có bước tiến vượt bậc – với màn hình cảm ứng lớn, bộ xử lý mạnh, bộ nhớ RAM dồi dào, cùng lúc đó là sự xuất hiện của iOS (Swift/Objective-C) và Android (Java/Kotlin) – lập trình viên mới thực sự có cơ hội khai phá sức mạnh của nền tảng di động một cách toàn diện. Giao diện đẹp hơn, tốc độ xử lý nhanh hơn và đặc biệt là hệ sinh thái phát triển (SDK, IDE, kho ứng dụng) đầy đủ và nhất quán, đã tạo ra một cuộc cách mạng trong phát triển ứng dụng.
\end{flushleft}

% 2.7
\subsection{Thị trường di động trên toàn thế giới}
\renewcommand{\labelitemi}{--}    
\begin{flushleft}
  \hspace*{0.8cm}Trước năm 2010, khi khái niệm “smartphone” còn khá mới mẻ với phần lớn người dùng, thị trường điện thoại di động toàn cầu vẫn được dẫn dắt bởi những thương hiệu gắn liền với sự bền bỉ, phổ thông và dễ sử dụng. Nổi bật trong số đó, không ai khác chính là Nokia – “ông vua” không ngai của thời kỳ tiền smartphone.
  \end{flushleft}
  
  \begin{flushleft}
  \hspace*{0.8cm}Nokia đã từng bước định hình trải nghiệm di động cho hàng triệu người dùng trên toàn thế giới, từ những chiếc điện thoại đen trắng cơ bản đến các dòng máy màu, hỗ trợ nhạc chuông đa âm, chơi game Java, cài đặt phần mềm và chụp ảnh. Những model như Nokia 1100, 6300, N70, N95 không chỉ phổ biến mà còn trở thành biểu tượng của một thời kỳ công nghệ giản đơn mà hiệu quả.
  \end{flushleft}
  
  \begin{flushleft}
  \hspace*{0.8cm}Tại Việt Nam, Nokia chiếm trọn niềm tin của người dùng nhờ thiết kế chắc chắn, pin “trâu”, giá thành hợp lý và khả năng hoạt động bền bỉ trong nhiều năm. Sự phổ biến của thương hiệu này mạnh đến mức tên “Nokia” từng được dùng để gọi chung cho bất kỳ chiếc điện thoại nào, dù có thể đó là sản phẩm của hãng khác.
  \end{flushleft}
  
  \begin{flushleft}
  \hspace*{0.8cm}Song song với Nokia, các hãng như Blackberry, Sony Ericsson, Samsung, HTC cũng không ngừng nỗ lực chiếm lĩnh thị phần bằng cách phát triển các hệ điều hành riêng biệt hoặc hợp tác với bên thứ ba để tạo ra sự khác biệt. Một số nền tảng phổ biến lúc bấy giờ gồm:
  \begin{itemize}
  \item Symbian (Nokia)
  \item Blackberry OS (Blackberry)
  \item Windows Mobile, Windows CE (Microsoft)
  \item Palm OS, Linux Mobile, Bada OS (Samsung)
  \end{itemize}
  \end{flushleft}
  
  \begin{flushleft}
  \hspace*{0.8cm}Tuy nhiên, một điểm yếu chung của tất cả các hệ điều hành kể trên chính là sự thiếu thống nhất và không thân thiện với lập trình viên. Mỗi hệ điều hành lại có giao diện riêng, API riêng, ngôn ngữ lập trình và tài liệu khác nhau – gây ra vô vàn khó khăn cho nhà phát triển:
  \begin{itemize}
  \item Trải nghiệm người dùng không đồng đều, giao diện không thống nhất.
  \item Bộ công cụ phát triển (SDK) còn sơ khai, thiếu tính năng và khó tiếp cận.
  \item Không có kho ứng dụng tập trung, dẫn đến việc phân phối phần mềm cực kỳ phức tạp.
  \item[]Điều này khiến phần lớn lập trình viên thời điểm đó vẫn còn thờ ơ hoặc làm việc theo kiểu "cho có", bởi chưa có hệ sinh thái đủ hấp dẫn để đầu tư nghiêm túc.
  \end{itemize}
  \end{flushleft}
  
  \begin{flushleft}
  \hspace*{0.8cm}Mọi thứ chỉ thật sự thay đổi khi Apple chính thức ra mắt iPhone vào năm 2007. Thiết bị này không chỉ là một chiếc điện thoại thông minh đầu tiên theo đúng nghĩa hiện đại, mà còn là bước ngoặt vĩ đại của toàn ngành công nghiệp di động.
  \end{flushleft}
  
  \begin{flushleft}
  \hspace*{0.8cm}Với màn hình cảm ứng đa điểm, thiết kế tinh tế không dùng bàn phím vật lý, hiệu năng mạnh mẽ và hệ điều hành ổn định, iPhone đã đặt ra một chuẩn mực mới cho smartphone. Tuy nhiên, cú hích lớn nhất chỉ đến vào năm 2008, khi App Store chính thức ra đời – nền tảng phân phối ứng dụng tập trung đầu tiên trong lịch sử di động.
  \end{flushleft}
  
  \begin{flushleft}
  \hspace*{0.8cm}App Store không chỉ giúp người dùng dễ dàng tìm kiếm và cài đặt phần mềm, mà còn mở ra cơ hội kinh doanh ứng dụng chuyên nghiệp cho lập trình viên trên toàn thế giới. Với một bộ công cụ phát triển mạnh mẽ, tài liệu hướng dẫn đầy đủ và cộng đồng sôi nổi, iOS SDK đã biến lập trình di động từ một công việc khó khăn, mơ hồ thành một ngành công nghiệp sáng tạo thực thụ.
  \end{flushleft}
  
  \begin{flushleft}
  \hspace*{0.8cm}Kể từ đây, lập trình ứng dụng di động bước sang một kỷ nguyên mới, nơi trải nghiệm người dùng, hiệu năng và hệ sinh thái phát triển đóng vai trò trung tâm. Cuộc đua giữa các nền tảng như iOS và Android cũng bắt đầu, tạo nên một trong những giai đoạn sôi động nhất trong lịch sử công nghệ di động.
  \end{flushleft}

% 2.8
\subsection{Thị trường và sự cạnh tranh trong ngành lập trình di động}
\renewcommand{\labelitemi}{--}    
    \begin{flushleft}
        \hspace*{0.8cm}Khi thị trường thiết bị di động bắt đầu bùng nổ vào đầu thập niên 2010, ba ông lớn công nghệ là Microsoft, Apple và Google đã nhanh chóng bước vào một cuộc cạnh tranh gay gắt nhằm chiếm lĩnh thị phần trong cả phần cứng, hệ điều hành và hệ sinh thái ứng dụng. Mỗi công ty lựa chọn một chiến lược phát triển riêng biệt, qua đó không chỉ định hình diện mạo ngành công nghiệp di động mà còn tác động sâu rộng đến cách thức phát triển ứng dụng di động sau này.
    \end{flushleft}

    \begin{flushleft}
      \hspace*{0.8cm}Trong ba ông lớn, Microsoft là cái tên bước vào cuộc chơi di động với nhiều tham vọng nhưng lại chậm chân. Sau khi nhận ra xu thế bùng nổ của smartphone, Microsoft đã tiến hành một thương vụ lớn vào năm 2013: mua lại bộ phận thiết bị và dịch vụ của Nokia với kỳ vọng tạo ra một hệ sinh thái Windows toàn diện, thống nhất từ máy tính, máy tính bảng đến điện thoại. Triết lý phát triển "Windows Universal Platform" (UWP) – một ứng dụng chạy được trên mọi thiết bị – là nỗ lực nhằm tối ưu trải nghiệm lập trình viên và người dùng. Tuy nhiên, Windows Phone lại xuất hiện quá muộn so với iOS và Android. Kho ứng dụng nghèo nàn, cộng đồng lập trình viên thiếu mặn mà, và mặc dù hệ điều hành này có trải nghiệm mượt mà, nhưng lại thiếu linh hoạt và không thể cá nhân hóa sâu. Kết quả, Windows Phone dần bị người dùng quay lưng và cuối cùng bị khai tử, để lại một bài học sâu sắc về tầm quan trọng của thời điểm gia nhập thị trường và sự linh hoạt trong thích nghi với xu thế công nghệ.
    \end{flushleft}

    \begin{flushleft}
      \hspace*{0.8cm}Ngược lại với Microsoft, Apple là công ty dẫn đầu trong cả trải nghiệm người dùng lẫn xây dựng hệ sinh thái khép kín. Với triết lý thiết kế "từ trong ra ngoài" – nghĩa là kiểm soát cả phần cứng và phần mềm – Apple tạo ra sự đồng bộ, ổn định và an toàn cao cho các thiết bị của mình. iPhone nổi bật với thiết kế tinh tế, hiệu năng mượt mà; hệ điều hành iOS ít lỗi và có tính bảo mật cao; App Store được quản lý chặt chẽ, đảm bảo chất lượng ứng dụng. Đồng thời, Apple luôn chú trọng đến việc hỗ trợ lập trình viên thông qua các công cụ chuyên nghiệp như Xcode, ngôn ngữ Swift và các framework như UIKit hay SwiftUI. Sự nhất quán này giúp Apple duy trì vị thế hàng đầu trong phân khúc smartphone cao cấp và sở hữu lượng người dùng trung thành cao bậc nhất thế giới.
    \end{flushleft}

    \begin{flushleft}
      \hspace*{0.8cm}Về phía Google, chiến lược phát triển của họ xoay quanh triết lý mã nguồn mở, với Android là trung tâm của hệ sinh thái di động. Thay vì sản xuất phần cứng đại trà như Apple, Google phát triển Android như một nền tảng mở, cho phép các nhà sản xuất như Samsung, Xiaomi, Oppo hay Huawei tùy biến và tích hợp hệ điều hành này vào thiết bị của họ. Cách tiếp cận này giúp Android nhanh chóng chiếm lĩnh thị phần toàn cầu. Ban đầu, Android bị phê phán vì tính phân mảnh và kém ổn định, nhưng theo thời gian, Google đã liên tục cải tiến hệ điều hành qua từng phiên bản, siết chặt tiêu chuẩn chất lượng trên Google Play và giới thiệu dòng thiết bị Pixel nhằm kiểm soát tốt hơn trải nghiệm người dùng. Nhờ vào sức mạnh về dữ liệu, trí tuệ nhân tạo và mạng lưới dịch vụ (Gmail, Maps, YouTube...), Google vẫn giữ vững vị thế là trụ cột của ngành di động hiện đại.
    \end{flushleft}

    \begin{flushleft}
      \hspace*{0.8cm}Sự cạnh tranh giữa Microsoft, Apple và Google – ba công ty với ba con đường phát triển khác nhau – không chỉ thúc đẩy đổi mới công nghệ mà còn tạo ra một hệ sinh thái lập trình di động phong phú, đa dạng và không ngừng phát triển. Chính sự đối đầu chiến lược này đã đặt nền móng cho những xu hướng lập trình đa nền tảng ngày nay.
    \end{flushleft}

% 2.9
\subsection{Hai hệ điều hành phổ biến nhất ngày nay: iOS và Android.}
\renewcommand{\labelitemi}{--}   
\begin{flushleft}
  \hspace*{0.8cm}Trong thị trường thiết bị di động hiện nay, iOS và Android là hai hệ điều hành chiếm lĩnh tuyệt đối, đồng thời là nền tảng chủ đạo cho việc phát triển ứng dụng di động trên toàn cầu. Mỗi hệ điều hành mang trong mình triết lý thiết kế, kiến trúc kỹ thuật và hệ sinh thái lập trình riêng, qua đó ảnh hưởng mạnh mẽ đến lựa chọn công cụ và quy trình làm việc của lập trình viên.
\end{flushleft}
    \begin{flushleft}
        \hspace*{0.8cm}iOS là hệ điều hành được phát triển độc quyền bởi Apple, chủ yếu sử dụng trên các thiết bị như iPhone, iPad và iPod Touch. Đặc điểm nổi bật của iOS là sự ổn định, hiệu suất cao và mức độ bảo mật nghiêm ngặt, được đảm bảo bởi quy trình kiểm duyệt ứng dụng chặt chẽ trên App Store. Về mặt lập trình, trước đây iOS sử dụng Objective-C, nhưng kể từ năm 2014, ngôn ngữ Swift – hiện đại, dễ học và tối ưu hóa hiệu năng – đã dần thay thế và trở thành ngôn ngữ chính trong phát triển iOS. Các công cụ hỗ trợ như Xcode đóng vai trò trung tâm trong quá trình phát triển, với khả năng mô phỏng, kiểm thử và tối ưu hóa ứng dụng. Ngoài ra, Apple cung cấp nhiều framework mạnh mẽ như UIKit và SwiftUI để xây dựng giao diện người dùng một cách linh hoạt. Tuy nhiên, do hệ sinh thái đóng, lập trình iOS thường phụ thuộc hoàn toàn vào công cụ và quy định của Apple.
    \end{flushleft}

    \begin{flushleft}
      \hspace*{0.8cm}Android, trái lại, là một hệ điều hành mã nguồn mở do Google phát triển, dựa trên nền tảng Linux. Sự cởi mở trong kiến trúc giúp Android được sử dụng trên hàng trăm thương hiệu thiết bị khác nhau như Samsung, Sony, Xiaomi, hay Oppo, từ đó tạo ra một thị trường đa dạng và cạnh tranh mạnh mẽ. Về ngôn ngữ lập trình, Android ban đầu sử dụng Java, nhưng từ năm 2017, Kotlin đã được Google công nhận là ngôn ngữ chính thức do khả năng viết mã ngắn gọn, an toàn và tương thích tốt với Java. Android Studio là môi trường phát triển tích hợp chủ đạo, cung cấp nhiều công cụ hữu ích như trình giả lập thiết bị, Gradle để quản lý dự án, và Firebase để hỗ trợ backend. Nhờ tính chất mở, Android cho phép các nhà sản xuất tùy chỉnh sâu hệ điều hành, nhưng điều này cũng dẫn đến sự phân mảnh trong trải nghiệm người dùng và thách thức trong kiểm thử ứng dụng.
    \end{flushleft}

    \begin{flushleft}
      \hspace*{0.8cm}Bên cạnh phát triển ứng dụng gốc (native) cho từng hệ điều hành, ngày càng nhiều lập trình viên lựa chọn các công cụ phát triển đa nền tảng để tiết kiệm thời gian và công sức. Các nền tảng như Flutter (Google), React Native (Meta), Xamarin (Microsoft) và Unity (phổ biến trong phát triển game) cho phép viết một lần, chạy trên nhiều hệ điều hành. Cụ thể, Flutter sử dụng ngôn ngữ Dart và nổi bật với hiệu suất mượt mà cùng giao diện đẹp; React Native dựa vào JavaScript, cho phép tái sử dụng logic ứng dụng từ web sang mobile; Xamarin dùng ngôn ngữ C\# để tạo ứng dụng native cho cả iOS lẫn Android; còn Unity, vốn mạnh trong phát triển trò chơi, cũng có thể tạo ra các ứng dụng tương tác cao. Mặc dù các công cụ này mang lại lợi ích về năng suất, nhưng vẫn tồn tại một số hạn chế về hiệu năng và khả năng truy cập đến các API hệ thống sâu.
    \end{flushleft}

    \begin{flushleft}
      \hspace*{0.8cm}Tóm lại, sự phát triển vượt bậc của iOS và Android, cùng với sự ra đời của các công cụ phát triển đa nền tảng, đã tạo nên một môi trường phong phú, năng động và đầy tiềm năng cho ngành lập trình di động. Việc hiểu rõ đặc điểm của từng hệ điều hành và lựa chọn công cụ phù hợp sẽ là yếu tố then chốt giúp lập trình viên tối ưu hóa hiệu quả và chất lượng sản phẩm.
    \end{flushleft}
\section{Tổng quan về các kiến trúc phần mềm}

\begin{flushleft}
  \hspace*{0.8cm}Trong lĩnh vực phát triển ứng dụng di động, kiến trúc phần mềm đóng vai trò vô cùng quan trọng. Việc lựa chọn mô hình kiến trúc phù hợp giúp lập trình viên tổ chức mã nguồn một cách hợp lý, dễ bảo trì và mở rộng về sau. Hai hệ điều hành phổ biến nhất hiện nay – iOS và Android – tuy có nhiều điểm khác nhau, nhưng đều khuyến khích sử dụng các mẫu kiến trúc phần mềm để nâng cao hiệu quả phát triển ứng dụng.
\end{flushleft}

% 3.1
\subsection{Mẫu kiến trúc Layers}
\renewcommand{\labelitemi}{--}    
    \begin{flushleft}
        \hspace*{0.8cm}Cả hai hệ điều hành đều cổ vũ lập trình viên áp dụng một số mẫu kiến trúc chung nhằm đảm bảo cấu trúc rõ ràng, dễ kiểm soát. Một trong những mẫu phổ biến là Layers (phân lớp). Theo mô hình này, phần mềm được chia thành ba lớp chính:
        \setlength{\leftmargini}{1.5cm}
        \begin{itemize}
            \item Presentation Layer (Lớp giao diện): Giao tiếp trực tiếp với người dùng, hiển thị thông tin và nhận tương tác.
            \item Business Logic Layer (Lớp xử lý nghiệp vụ): Chứa logic của ứng dụng, xử lý các thao tác từ người dùng, thực hiện tính toán, kiểm tra dữ liệu.
            \item Data Layer (Lớp dữ liệu): Quản lý dữ liệu, truy xuất từ cơ sở dữ liệu nội bộ hoặc từ máy chủ.
        \end{itemize}
    \end{flushleft}

    \begin{flushleft}
      \hspace*{0.8cm}$\Rightarrow$ Ưu điểm của mô hình này là dễ bảo trì, dễ kiểm thử, giảm sự phụ thuộc giữa các thành phần. Mỗi lớp chỉ thực hiện một nhiệm vụ cụ thể, giúp mã nguồn rõ ràng và dễ quản lý, đặc biệt trong các dự án lớn.
    \end{flushleft}

% 3.2
\subsection{Mẫu kiến trúc MVC}
\renewcommand{\labelitemi}{--}    
    \begin{flushleft}
        \hspace*{0.8cm}Trên nền tảng iOS, MVC là mô hình kiến trúc lâu đời và phổ biến. Các lập trình viên iOS thường ưa chuộng mô hình này hơn. Nó phân chia phần mềm thành ba thành phần chính:
        \setlength{\leftmargini}{1.5cm}
        \begin{itemize}
            \item Model: Quản lý dữ liệu, logic xử lý và các quy tắc nghiệp vụ.
            \item View: Hiển thị thông tin cho người dùng, như giao diện ứng dụng.
            \item Controller: Là cầu nối giữa Model và View. Nó nhận dữ liệu từ Model và cập nhật cho View, đồng thời xử lý các tương tác từ người dùng.
        \end{itemize}
    \end{flushleft}

    \begin{flushleft}
      \hspace*{0.8cm}$\Rightarrow$ Mô hình này đơn giản, dễ hiểu và dễ áp dụng, đặc biệt với những người mới bắt đầu học lập trình iOS. Tuy nhiên, một vấn đề lớn thường gặp trong MVC của iOS là Controller thường bị "phình to", xử lý quá nhiều logic, khiến mã khó đọc, khó bảo trì. Đây được gọi là hiện tượng “Massive View Controller” – một trong những lý do khiến nhiều lập trình viên hiện đại tìm đến các mô hình thay thế như MVVM hoặc VIPER.
    \end{flushleft}

% 3.3
\subsection{Mô hình kiến trúc MVVM}
\renewcommand{\labelitemi}{--}    
    \begin{flushleft}
        \hspace*{0.8cm}Khác với nền tảng iOS,  nền tảng Android lại ưu tiên sử dụng mô hình MVVM, phù hợp hơn với kiến trúc hiện đại của Android và các công cụ hỗ trợ như LiveData, ViewModel, và Data Binding.
        \setlength{\leftmargini}{1.5cm}
        \begin{itemize}
            \item Model: Cũng giống kiến trúc MVC, thành phần này quản lý dữ liệu và logic xử lý.
            \item View: Giao diện người dùng, hiển thị dữ liệu từ ViewModel.
            \item ViewModel: Trung gian giữa View và Model. Nó không chứa tham chiếu trực tiếp đến View, nhưng nhờ cơ chế observer pattern (theo dõi), dữ liệu thay đổi trong ViewModel sẽ được cập nhật tự động lên View.
        \end{itemize}
    \end{flushleft}
    \begin{flushleft}
      \hspace*{0.8cm}$\Rightarrow$ MVVM giúp giảm mạnh sự phụ thuộc giữa UI và logic nghiệp vụ. Nhờ đó, View trở nên “mỏng”, chỉ đảm nhiệm việc hiển thị, trong khi mọi logic đều được xử lý trong ViewModel. Điều này giúp dễ kiểm thử, tái sử dụng code và phát triển theo nhóm.
    \end{flushleft}

% 3.4
\subsection{Hỗ trợ mô hình Client – Server}
\renewcommand{\labelitemi}{--}    
    \begin{flushleft}
        \hspace*{0.8cm}Bên cạnh kiến trúc bên trong ứng dụng, cả iOS và Android đều hỗ trợ xây dựng ứng dụng dựa trên mô hình Client – Server.
        \setlength{\leftmargini}{1.5cm}
        \begin{itemize}
            \item Ứng dụng hoạt động như Client, gửi các yêu cầu HTTP (GET, POST, PUT, DELETE…) đến một Server.
            \item Server xử lý yêu cầu, truy xuất dữ liệu từ cơ sở dữ liệu, sau đó phản hồi về Client dưới dạng dữ liệu JSON hoặc XML.
        \end{itemize}
    \end{flushleft}

    \begin{flushleft}
      \hspace*{0.8cm}Cả hai nền tảng đều cung cấp thư viện hỗ trợ như:
      \setlength{\leftmargini}{1.5cm}
      \begin{itemize}
          \item iOS: URLSession, Alamofire
          \item Android: Retrofit, OkHttp
      \end{itemize}
    \end{flushleft}

    \begin{flushleft}
      \hspace*{0.8cm}$\Rightarrow$ Mô hình Client – Server tạo điều kiện cho ứng dụng hoạt động linh hoạt, nhẹ, dễ cập nhật và đồng bộ dữ liệu giữa nhiều thiết bị người dùng khác nhau.
    \end{flushleft}

% 3.5
\subsection{Đánh giá tổng quan}
\renewcommand{\labelitemi}{--}    
    \begin{flushleft}
        \hspace*{0.8cm}Nhìn chung, dù cùng hướng tới việc xây dựng phần mềm có cấu trúc rõ ràng và dễ bảo trì, iOS và Android lại có những lựa chọn mô hình kiến trúc khác nhau. iOS truyền thống với MVC, tuy đơn giản nhưng dễ bị quá tải ở phần Controller, trong khi Android hiện đại hóa với MVVM, tách biệt rõ ràng các thành phần, tận dụng được sức mạnh của các thư viện hỗ trợ. Cả hai nền tảng cũng đều hỗ trợ trao đổi dữ liệu theo mô hình Client – Server, phù hợp với nhu cầu xây dựng ứng dụng kết nối mạng ngày nay. Việc hiểu rõ ưu – nhược điểm của từng mô hình sẽ giúp lập trình viên lựa chọn giải pháp kiến trúc phù hợp, nâng cao hiệu quả phát triển ứng dụng.
    \end{flushleft}

\section{Phân tích chi tiết các kiến trúc phần mềm}

\begin{flushleft}
  \hspace*{0.8cm}Ba mô hình kiến trúc phần mềm phổ biến là kiến trúc ba tầng, MVC và MVVM đều hướng đến việc tổ chức mã nguồn rõ ràng, dễ bảo trì và mở rộng. Kiến trúc ba tầng chia hệ thống thành ba phần riêng biệt: trình diễn, nghiệp vụ và dữ liệu, giúp tăng tính độc lập giữa các tầng và phù hợp với các hệ thống có quy mô lớn. Trong khi đó, MVC tập trung vào việc tách biệt giao diện (View), dữ liệu (Model) và điều khiển (Controller), thường được sử dụng trong phát triển ứng dụng iOS với lợi thế trong việc hỗ trợ nhiều nhóm làm việc song song. MVVM, một phiên bản hiện đại hơn, thay Controller bằng ViewModel – giúp kết nối thông minh giữa View và Model thông qua Data Binding. Nhờ vậy, MVVM giảm sự phụ thuộc giữa các thành phần, tăng khả năng tự động hóa trong cập nhật giao diện và đặc biệt phù hợp với phát triển ứng dụng Android. Mỗi kiến trúc đều có ưu điểm riêng và được lựa chọn tùy thuộc vào nền tảng phát triển, quy mô hệ thống và yêu cầu kỹ thuật cụ thể.
  \end{flushleft}
  
  

% 4.1
\subsection{Kiến trúc phần mềm ba tầng (Three-tier architecture)}
\renewcommand{\labelitemi}{--}    
    \begin{flushleft}
        \hspace*{0.8cm}Trong các mô hình thiết kế phần mềm hiện đại, kiến trúc ba tầng được xem là một phương pháp tổ chức phần mềm hiệu quả, đặc biệt phù hợp với các hệ thống có quy mô lớn và yêu cầu cao về khả năng bảo trì cũng như mở rộng.
    \end{flushleft}

    \begin{flushleft}
      \hspace*{0.8cm}Kiến trúc này được chia thành ba tầng riêng biệt: trình diễn (Presentation), nghiệp vụ (Business Logic) và dữ liệu (Data).
      Mỗi tầng có một vai trò cụ thể, tách biệt nhưng có mối liên hệ chặt chẽ với nhau nhằm đảm bảo tính ổn định, dễ dàng phát triển cũng như tái sử dụng mã nguồn.
    \end{flushleft}

    \begin{flushleft}
      \hspace*{0.8cm}Ở tầng đầu tiên, tầng trình diễn là nơi tiếp xúc trực tiếp với người dùng.
      Tầng trình diễn chịu trách nhiệm hiển thị thông tin một cách rõ ràng, trực quan thông qua giao diện đồ họa, từ đó hỗ trợ người dùng tương tác dễ dàng với hệ thống.
    \end{flushleft}

    \begin{flushleft}
      \hspace*{0.8cm}Thông tin được hiển thị ở tầng này thường được lấy từ tầng nghiệp vụ đã xử lý sẵn, điều đó giúp giao diện luôn giữ được sự đơn giản và dễ thay đổi.
    \end{flushleft}

    \begin{flushleft}
      \hspace*{0.8cm}Việc không xử lý logic nghiệp vụ tại tầng giao diện mang lại lợi ích lớn trong việc tái sử dụng mã nguồn giao diện cho nhiều ngữ cảnh khác nhau, chẳng hạn như khi chuyển đổi giữa phiên bản web và mobile mà không cần thay đổi logic cốt lõi.
    \end{flushleft}

    \begin{flushleft}
      \hspace*{0.8cm}Tiếp theo là tầng nghiệp vụ, tầng trung tâm và đóng vai trò quan trọng nhất trong toàn bộ kiến trúc.
    \end{flushleft}

    \begin{flushleft}
      \hspace*{0.8cm}Tầng nghiệp vụ là nơi hiện thực các quy tắc, quy trình và logic nghiệp vụ của ứng dụng.
    \end{flushleft}

    \begin{flushleft}
      \hspace*{0.8cm}Mọi yêu cầu từ tầng trình diễn sẽ được tầng nghiệp vụ tiếp nhận, xử lý và gửi tới tầng dữ liệu khi cần.
    \end{flushleft}

    \begin{flushleft}
      \hspace*{0.8cm}Ngược lại, khi dữ liệu được truy xuất từ tầng dữ liệu, tầng nghiệp vụ sẽ định dạng lại hoặc xử lý tiếp trước khi gửi trả về cho tầng giao diện.
      Điểm mạnh của tầng nghiệp vụ nằm ở khả năng cô lập các thao tác phức tạp và kiểm soát luồng thông tin, từ đó giúp nâng cao khả năng kiểm thử, bảo trì và phát triển mở rộng hệ thống trong tương lai.
    \end{flushleft}

    \begin{flushleft}
      \hspace*{0.8cm}Cuối cùng là tầng dữ liệu, nơi lưu trữ toàn bộ thông tin cần thiết của hệ thống.
    \end{flushleft}

    \begin{flushleft}
      \hspace*{0.8cm}Tầng dữ liệu đảm nhận nhiệm vụ lưu trữ, truy vấn, cập nhật và bảo vệ dữ liệu thông qua các hệ quản trị cơ sở dữ liệu như SQL hoặc NoSQL.
    \end{flushleft}

    \begin{flushleft}
      \hspace*{0.8cm}Tại tầng này, hệ thống có thể thực hiện các thao tác tối ưu hóa truy vấn, quản lý kết nối đến cơ sở dữ liệu từ xa hoặc tích hợp với các dịch vụ lưu trữ đám mây.
    \end{flushleft}

    \begin{flushleft}
      \hspace*{0.8cm}Tầng dữ liệu càng được tối ưu thì hiệu suất toàn hệ thống càng cao, đặc biệt với các ứng dụng có khối lượng dữ liệu lớn và nhu cầu xử lý đồng thời cao.
    \end{flushleft}

    \begin{flushleft}
      \hspace*{0.8cm}Tổng thể, kiến trúc ba tầng mang lại nhiều lợi ích vượt trội như dễ bảo trì, dễ mở rộng, dễ kiểm thử và thuận lợi cho phát triển theo nhóm.
    \end{flushleft}

    \begin{flushleft}
      \hspace*{0.8cm}Tổng thể, kiến trúc ba tầng mang lại nhiều lợi ích vượt trội như dễ bảo trì, dễ mở rộng, dễ kiểm thử và thuận lợi cho phát triển theo nhóm.
    \end{flushleft}

    \begin{flushleft}
      \hspace*{0.8cm}
      Việc phân tách rõ vai trò của từng tầng không chỉ giúp quá trình phát triển trở nên mạch lạc và dễ quản lý mà còn giúp ứng dụng thích nghi tốt với các thay đổi trong tương lai.
    \end{flushleft}

    \begin{flushleft}
      \hspace*{0.8cm}Chính vì thế, kiến trúc ba tầng đã và đang được sử dụng rộng rãi trong nhiều hệ thống phần mềm từ ứng dụng nội bộ, ứng dụng thương mại điện tử, đến hệ thống phân tán quy mô lớn.
    \end{flushleft}

% 4.2
\subsection{Kiến trúc MVC (Model – View – Controller)}
\renewcommand{\labelitemi}{--}    
    \begin{flushleft}
        \hspace*{0.8cm}Trong quá trình phát triển phần mềm, đặc biệt là trong các ứng dụng di động trên nền tảng iOS, kiến trúc MVC (Model – View – Controller) đóng vai trò như một nền tảng tổ chức mã nguồn kinh điển, giúp lập trình viên tách biệt các chức năng chính của ứng dụng thành ba phần riêng biệt.
    \end{flushleft}

    \begin{flushleft}
      \hspace*{0.8cm}Việc phân chia này không chỉ nhằm giảm độ phức tạp của hệ thống mà còn tạo điều kiện thuận lợi cho quá trình bảo trì, kiểm thử và mở rộng về sau. Theo mô hình này, toàn bộ hệ thống được chia thành ba thành phần chính: Model (xử lý dữ liệu và logic nghiệp vụ), View (hiển thị giao diện người dùng), và Controller (điều phối luồng xử lý giữa Model và View). Mỗi thành phần trong ba thành phần này đảm nhiệm một vai trò cụ thể và độc lập, tuy nhiên lại phối hợp chặt chẽ để tạo nên một hệ thống vận hành ổn định và linh hoạt.
    \end{flushleft}

    \begin{flushleft}
      \hspace*{0.8cm}Với vai trò là trung tâm lưu trữ dữ liệu, thành phần Model trong kiến trúc MVC chịu trách nhiệm định nghĩa các đối tượng dữ liệu, quản lý logic nghiệp vụ và thực hiện các thao tác như truy xuất, cập nhật hoặc lưu trữ dữ liệu từ các nguồn như cơ sở dữ liệu hoặc API bên ngoài. Điều đáng chú ý là Model hoạt động hoàn toàn độc lập với giao diện hiển thị, tức là nó không biết và không quan tâm đến cách dữ liệu sẽ được trình bày ra sao trước người dùng, từ đó tạo ra tính tái sử dụng cao. Trong thực tế, một Model có thể được dùng chung cho nhiều giao diện khác nhau mà không cần thay đổi logic bên trong.
    \end{flushleft}

    \begin{flushleft}
      \hspace*{0.8cm}Ngược lại với Model, thành phần View lại tập trung vào việc trình bày thông tin ra bên ngoài, nơi người dùng có thể quan sát và tương tác trực tiếp với ứng dụng. View chịu trách nhiệm chuyển dữ liệu từ Model thành định dạng có thể hiển thị được, chẳng hạn như văn bản, hình ảnh hay biểu đồ. Tuy nhiên, View trong MVC không bao giờ xử lý logic nghiệp vụ mà chỉ đóng vai trò như một “màn hình trình diễn”, nơi phản ánh chính xác dữ liệu đã được xử lý bởi Model thông qua trung gian là Controller. Chính vì vậy, View có thể được thay thế, cập nhật hoặc thiết kế lại mà không ảnh hưởng đến logic bên trong của ứng dụng.
    \end{flushleft}

    \begin{flushleft}
      \hspace*{0.8cm}Trong khi đó, thành phần Controller giữ vai trò như một “bộ não” điều phối hoạt động của toàn bộ ứng dụng. Mỗi khi người dùng thực hiện một hành động nào đó trên View, chẳng hạn như nhấn nút, nhập văn bản hoặc vuốt màn hình, Controller sẽ tiếp nhận sự kiện đó và quyết định phải thực hiện những thao tác gì tiếp theo. Điều này có thể bao gồm việc gọi đến các hàm xử lý trong Model, cập nhật trạng thái dữ liệu hoặc yêu cầu View hiển thị kết quả mới. Do giữ vai trò kết nối và kiểm soát hai thành phần còn lại, Controller thường là nơi tập trung phần lớn các quy tắc điều phối nghiệp vụ trong ứng dụng.
    \end{flushleft}

    \begin{flushleft}
      \hspace*{0.8cm}Một điểm mạnh lớn của kiến trúc MVC là khả năng tách biệt vai trò, điều này cho phép các lập trình viên làm việc song song trên cùng một dự án mà không gây ra xung đột. Ví dụ, người thiết kế giao diện có thể xây dựng View mà không cần hiểu chi tiết về logic xử lý, trong khi lập trình viên backend có thể phát triển Model một cách độc lập. Controller đóng vai trò là cầu nối giữa hai nhóm này, đảm bảo toàn bộ ứng dụng vận hành nhịp nhàng. Thêm vào đó, do Model và View hoạt động độc lập, nên khi có nhu cầu mở rộng ứng dụng hoặc thay đổi một phần giao diện, lập trình viên chỉ cần điều chỉnh View mà không phải can thiệp vào các phần xử lý bên trong, từ đó tiết kiệm thời gian và công sức phát triển. Tất cả những yếu tố đó khiến MVC vẫn là một lựa chọn phổ biến trong nhiều dự án iOS hiện đại, đặc biệt là khi sử dụng Xcode và Swift – hai công cụ hỗ trợ tốt cho mô hình này.
    \end{flushleft}

% 4.3
\subsection{Kiến trúc MVVM (Model - View - ViewModel)}
\renewcommand{\labelitemi}{--}    
    \begin{flushleft}
        \hspace*{0.8cm}Đối với các ứng dụng hiện đại, đặc biệt là trên nền tảng Android với sự hỗ trợ mạnh mẽ từ Jetpack và Kotlin, kiến trúc MVVM (Model – View – ViewModel) đã nhanh chóng trở thành một giải pháp thay thế lý tưởng cho MVC, nhờ khả năng giảm thiểu sự phụ thuộc giữa các thành phần và tự động hóa cập nhật giao diện thông qua cơ chế Data Binding. Mặc dù MVVM vẫn duy trì ba thành phần chính tương tự MVC, gồm Model, View và một thành phần trung gian, nhưng điểm khác biệt quan trọng nằm ở cách các thành phần này tương tác. Cụ thể, View không còn giao tiếp trực tiếp với Model như trong MVC, mà thay vào đó tương tác với ViewModel – một lớp trung gian thông minh có khả năng phản hồi tự động mỗi khi dữ liệu thay đổi, nhờ các cơ chế như LiveData hoặc Observable.
    \end{flushleft}

    \begin{flushleft}
      \hspace*{0.8cm}Trong MVVM, thành phần Model vẫn giữ nguyên vai trò như trong kiến trúc MVC, tức là nơi lưu trữ dữ liệu cốt lõi và thực hiện các thao tác xử lý nghiệp vụ như gọi API, truy vấn cơ sở dữ liệu hoặc tính toán logic phức tạp. Tuy nhiên, điểm khác biệt nằm ở chỗ Model không tương tác trực tiếp với giao diện người dùng, mà tất cả dữ liệu sẽ được trung chuyển qua ViewModel. Nhờ vậy, việc tách rời trách nhiệm được thực hiện triệt để hơn, giúp hệ thống trở nên dễ kiểm soát và dễ kiểm thử hơn.
    \end{flushleft}

    \begin{flushleft}
      \hspace*{0.8cm}Trong khi đó, View trong kiến trúc MVVM chỉ đóng vai trò như một công cụ hiển thị, tương tự như trong MVC, nhưng điểm mạnh vượt trội nằm ở khả năng liên kết dữ liệu hai chiều (two-way data binding). Khi một người dùng nhập dữ liệu vào một thành phần như EditText, sự thay đổi này sẽ ngay lập tức được cập nhật vào ViewModel. Ngược lại, nếu dữ liệu trong ViewModel thay đổi – chẳng hạn do Model vừa nhận dữ liệu mới từ server – thì View cũng sẽ được cập nhật một cách tự động mà không cần lập trình viên can thiệp thêm. Nhờ vào cơ chế tự động này, lượng mã xử lý sự kiện và cập nhật giao diện giảm đáng kể, đồng thời làm giảm nguy cơ xảy ra lỗi do cập nhật thủ công.
    \end{flushleft}

    \begin{flushleft}
      \hspace*{0.8cm}Thành phần trung gian – ViewModel – chính là điểm khác biệt lớn nhất và mạnh mẽ nhất trong kiến trúc MVVM. ViewModel không chỉ chịu trách nhiệm chuẩn bị và cung cấp dữ liệu cho View, mà còn đóng vai trò là nơi lưu trữ trạng thái giao diện, đảm bảo tính nhất quán giữa các phiên làm việc hoặc khi xoay màn hình. Thêm vào đó, ViewModel có thể sử dụng các đối tượng LiveData hoặc StateFlow để cho phép View quan sát sự thay đổi của dữ liệu theo thời gian thực. Điều này đặc biệt hữu ích trong các ứng dụng có giao diện phức tạp hoặc cần cập nhật liên tục, chẳng hạn như ứng dụng thương mại điện tử, mạng xã hội hoặc theo dõi vị trí.
    \end{flushleft}

    \begin{flushleft}
      \hspace*{0.8cm}Một lợi thế rõ rệt của MVVM so với MVC chính là mức độ giảm phụ thuộc giữa các thành phần, từ đó giúp hệ thống linh hoạt hơn khi mở rộng hoặc thay đổi một phần cụ thể. Do View và Model không còn liên kết trực tiếp, nên việc thay đổi cấu trúc giao diện hoặc định dạng dữ liệu trong Model sẽ không gây ảnh hưởng đến phần còn lại của hệ thống. Đồng thời, với sự hỗ trợ mạnh mẽ từ Jetpack (Android Architecture Components), việc triển khai MVVM trên Android trở nên thuận tiện hơn bao giờ hết nhờ các thư viện tích hợp như ViewModel, LiveData, Room và Data Binding.
    \end{flushleft}

    \begin{flushleft}
      \hspace*{0.8cm}Tóm lại, MVVM không chỉ là một bước tiến tự nhiên từ MVC mà còn là một kiến trúc phù hợp với xu hướng phát triển hiện đại – nơi giao diện cần phản hồi nhanh chóng, dữ liệu cập nhật liên tục và nhóm phát triển cần phối hợp hiệu quả. Nhờ khả năng tự động cập nhật giao diện khi dữ liệu thay đổi, giảm thiểu mã lặp, cũng như tăng khả năng kiểm thử và tái sử dụng mã nguồn, kiến trúc MVVM đã và đang trở thành lựa chọn ưu tiên trong các dự án Android có quy mô trung bình đến lớn, nơi yêu cầu về hiệu suất và khả năng mở rộng ngày càng được đặt lên hàng đầu.
    \end{flushleft}

% 4.4
\subsection{Kiến trúc Client/Server}
\renewcommand{\labelitemi}{--}    
    \begin{flushleft}
        \hspace*{0.8cm}Trong bối cảnh số hóa mạnh mẽ hiện nay, nơi mà hầu hết các ứng dụng đều cần tương tác và trao đổi dữ liệu với hệ thống bên ngoài thông qua Internet, kiến trúc Client/Server đã trở thành một nền tảng không thể thiếu trong thiết kế hệ thống phần mềm hiện đại. Đặc biệt, với các ứng dụng yêu cầu đồng bộ dữ liệu liên tục, cập nhật thông tin theo thời gian thực hoặc sử dụng tài nguyên được lưu trữ trên máy chủ từ xa, mô hình Client/Server đóng vai trò cốt lõi trong việc đảm bảo quá trình trao đổi dữ liệu diễn ra mượt mà, chính xác và có khả năng mở rộng cao. Tuy nhiên, để phát huy tối đa hiệu quả, kiến trúc Client/Server thường được sử dụng kết hợp với các kiến trúc nội bộ như MVC hoặc MVVM nhằm tách biệt rõ ràng giữa xử lý logic, giao diện người dùng và việc giao tiếp với hệ thống bên ngoài, từ đó giúp quá trình phát triển ứng dụng trở nên rõ ràng, có cấu trúc và dễ bảo trì hơn.
    \end{flushleft}

    \begin{flushleft}
      \hspace*{0.8cm}Về bản chất, kiến trúc Client/Server mô tả mô hình trong đó có sự phân chia rõ ràng giữa hai vai trò chính: Client (ứng dụng phía người dùng) và Server (máy chủ xử lý trung tâm). Trong mô hình này, các thiết bị đầu cuối như điện thoại, máy tính bảng hay trình duyệt web sẽ đóng vai trò là Client – nơi người dùng thực hiện các thao tác như gửi yêu cầu truy vấn dữ liệu, cập nhật thông tin cá nhân hoặc thao tác với giao diện ứng dụng. Những yêu cầu này, thường được gửi đi thông qua các giao thức phổ biến như HTTP, HTTPS hoặc WebSocket, sẽ được máy chủ tiếp nhận, xử lý và phản hồi tương ứng. Máy chủ, tức Server, sẽ tiếp nhận request từ Client, thực hiện các tác vụ cần thiết như truy xuất cơ sở dữ liệu, xử lý logic nghiệp vụ hoặc tính toán nâng cao, rồi trả lại kết quả phù hợp dưới dạng dữ liệu (thường là JSON hoặc XML).
    \end{flushleft}

    \begin{flushleft}
      \hspace*{0.8cm}Một điểm quan trọng trong kiến trúc Client/Server là tính phân tán và độc lập giữa hai thành phần. Trong khi Client tập trung vào trải nghiệm người dùng như hiển thị giao diện, điều hướng màn hình, thu thập và hiển thị dữ liệu, thì Server lại tập trung vào việc xử lý các logic phức tạp, đảm bảo tính nhất quán của dữ liệu, bảo mật truy cập, và cung cấp tài nguyên theo cách tối ưu nhất. Mối quan hệ giữa Client và Server là một mối quan hệ chặt chẽ nhưng độc lập: mỗi bên có thể được nâng cấp hoặc thay đổi mà không ảnh hưởng trực tiếp đến bên còn lại, miễn là giao diện lập trình ứng dụng (API) giữa hai bên vẫn được tuân thủ nhất quán. Nhờ sự phân chia nhiệm vụ rõ ràng này, hệ thống có thể dễ dàng mở rộng về quy mô, nâng cấp từng thành phần riêng biệt và tối ưu hiệu suất xử lý.
    \end{flushleft}

    \begin{flushleft}
      \hspace*{0.8cm}Kiến trúc Client/Server thể hiện rõ vai trò trung tâm trong nhiều loại ứng dụng hiện đại có quy mô người dùng lớn, chẳng hạn như mạng xã hội (Facebook, Instagram), ứng dụng ngân hàng (Vietcombank, BIDV SmartBanking), thương mại điện tử (Shopee, Tiki), hay các nền tảng học trực tuyến (Google Classroom, Coursera). Trong các ứng dụng này, Client đóng vai trò là công cụ trung gian cho người dùng truy cập dịch vụ, trong khi Server xử lý các yêu cầu như xác thực tài khoản, truy xuất đơn hàng, hoặc cung cấp nội dung bài giảng. Đặc biệt, với nhu cầu chia sẻ dữ liệu đồng bộ giữa nhiều thiết bị khác nhau, kiến trúc Client/Server cho phép dữ liệu luôn được lưu trữ tập trung trên máy chủ, đảm bảo tính nhất quán và khả năng truy cập mọi lúc, mọi nơi.
    \end{flushleft}

    \begin{flushleft}
      \hspace*{0.8cm}Tóm lại, kiến trúc Client/Server không chỉ đơn thuần là mô hình truyền thống trong phát triển phần mềm mà còn là nền tảng quan trọng trong xây dựng các hệ thống hiện đại có khả năng mở rộng và tương tác mạnh mẽ. Khi được kết hợp hợp lý với các kiến trúc nội bộ như MVC hoặc MVVM, kiến trúc này mang lại khả năng tổ chức mã nguồn rõ ràng, giảm độ phức tạp, đồng thời tối ưu hóa hiệu suất và khả năng bảo trì của hệ thống phần mềm, đặc biệt là trong các ứng dụng di động kết nối Internet như hiện nay.
    \end{flushleft}
\section{Các yếu tố ảnh hưởng đến chi phí phát triển ứng dụng}

% 5.1
\subsection{Các yếu tố chính:}
\renewcommand{\labelitemi}{--}    
    \begin{flushleft}
        \hspace*{0.8cm}Features and Functionality: Tính năng càng phức tạp, chi phí càng cao.
    \end{flushleft}

    \begin{flushleft}
      \hspace*{0.8cm}App Infrastructure: Liên quan đến cách tổ chức hệ thống backend, API, hệ thống lưu trữ, bảo mật, quản lý hiệu năng,...
    \end{flushleft}

    \begin{flushleft}
      \hspace*{0.8cm}App Complexity: Mức độ phức tạp về luồng xử lý, logic kinh doanh.
    \end{flushleft}

    \begin{flushleft}
      \hspace*{0.8cm}UX/UI Design: Thiết kế giao diện thân thiện, hiện đại cần đầu tư nhiều hơn.
    \end{flushleft}

    \begin{flushleft}
      \hspace*{0.8cm}App Maintenance: Chi phí bảo trì sau khi triển khai.
    \end{flushleft}

    \begin{flushleft}
      \hspace*{0.8cm}App Security: Yêu cầu bảo mật cao làm tăng chi phí.
    \end{flushleft}

    \begin{flushleft}
      \hspace*{0.8cm}App Category: Tùy theo ứng dụng thuộc lĩnh vực nào (y tế, tài chính, mạng xã hội...) sẽ có độ phức tạp khác nhau.
    \end{flushleft}

    \begin{flushleft}
      \hspace*{0.8cm}\# of Platforms \& Pages: Càng nhiều nền tảng (iOS/Android/Web), giao diện càng đa dạng → chi phí tăng.
    \end{flushleft}

    \begin{flushleft}
      \hspace*{0.8cm}Location of the Development Team: Địa điểm nhóm phát triển quyết định đến chi phí nhân công.
    \end{flushleft}

    \begin{flushleft}
      \hspace*{0.8cm}$\Rightarrow$ Vai trò then chốt của kiến trúc phần mềm - App Infrastructure (Hạ tầng ứng dụng):
      \setlength{\leftmargini}{1.5cm}
      \begin{itemize}
          \item Là xương sống của toàn bộ hệ thống phần mềm, quyết định khả năng mở rộng, hiệu suất, bảo mật, tích hợp dịch vụ khác (API, cơ sở dữ liệu, cloud...).
          \item Việc thiết kế hạ tầng tốt từ đầu giúp giảm chi phí bảo trì về sau và tăng độ ổn định của ứng dụng.
      \end{itemize}
    \end{flushleft}

    \begin{flushleft}
      \hspace*{0.8cm}
    \end{flushleft}

    \begin{flushleft}
      \hspace*{0.8cm}
    \end{flushleft}

% 
\subsection{Chi phí thuê nhân sự theo khu vực địa lý:}
\renewcommand{\labelitemi}{--}    
    \begin{flushleft}
        \hspace*{0.8cm}Bảng trên thể hiện khoản chi phí thuê theo giờ cho các vai trò chuyên môn trong ngành công nghệ thông tin, dựa trên thống kê từ 4 khu vực là United States (Hoa Kỳ), Latin America (Mỹ Latin), Eastern Europe (Đông Âu) và Asia (Châu Á). Đồng thời bao gồm 12 vai trò phổ biến trong các dự án phần mềm.
    \end{flushleft}

    \begin{flushleft}
      \hspace*{0.8cm}Nhóm Quản lý và Thiết kế Kiến trúc:
      \setlength{\leftmargini}{1.5cm}
      \begin{itemize}
          \item Nhóm này bao gồm 3 vai trò Business Analyst, Architect và Project Manager.
          \item Đây là nhóm có mức phí cao nhất, đặc biệt tại Hoa Kỳ (lên đến gần \$300/giờ).
          \item Mức giá giảm dần theo khu vực: US > Latin America > Eastern Europe > Asia
          \item Châu Á có mức phí thấp nhất, từ \$30 – \$48/giờ, khiến khu vực này hấp dẫn với các công ty thuê ngoài (outsourcing).
          \item[]$\Rightarrow$ Đây là các vị trí quyết định kiến trúc phần mềm, quy trình quản lý và kết nối giữa yêu cầu kinh doanh với kỹ thuật, nên đòi hỏi kinh nghiệm và kỹ năng cao.
      \end{itemize}
    \end{flushleft}

    \begin{flushleft}
      \hspace*{0.8cm}Nhóm Lập trình viên (Developer):
      \setlength{\leftmargini}{1.5cm}
      \begin{itemize}
          \item Nhóm này được chia theo kinh nghiệm của lập trình viên, bao gồm Jr. Developer, Mid-Level Dev, Sr. Developer và Lead Developer
          \item Mức lương tăng dần theo cấp bậc từ Jr. → Mid → Sr. → Lead.
          \item Sr. Developer và Lead Developer có chi phí gần tiệm cận nhóm kiến trúc sư tại một số khu vực.
          \item Asia tiếp tục là nơi có mức chi phí thấp nhất ở mọi cấp độ.
      \end{itemize}
    \end{flushleft}

    \begin{flushleft}
      \hspace*{0.8cm}Nhóm Kiểm thử phần mềm (QA):
      \setlength{\leftmargini}{1.5cm}
      \begin{itemize}
          \item Nhóm này cũng được chia dựa trên kinh nghiệm, bao gồm Junior QA, Mid-Level QA và Senior QA.
          \item QA ít tốn kém hơn Developer hoặc Architect, nhưng vẫn có mức chênh lệch đáng kể giữa các khu vực.
          \item Sr. QA tại Mỹ có thể lên đến \$169/giờ, cao hơn cả Mid-Level Dev tại Mỹ (\$140).
      \end{itemize}
    \end{flushleft}

    \begin{flushleft}
      \hspace*{0.8cm}Thiết kế giao diện (Graphic Designer):
      \setlength{\leftmargini}{1.5cm}
      \begin{itemize}
          \item Vai trò thiết kế tuy không chuyên sâu kỹ thuật nhưng vẫn chiếm chi phí tương đối, đặc biệt ở các thị trường phát triển như Mỹ.
      \end{itemize}
    \end{flushleft}



