
\section{Tổng quan về kiến trúc hệ thống}
\begin{flushleft}
    \hspace*{0.8cm}Kiến trúc hệ thống là nền tảng quan trọng trong phát triển ứng dụng di động, nơi các thành phần như thiết bị, ứng dụng và server phối hợp hoạt động. Trong phần này, người đọc sẽ tìm hiểu về vai trò của kiến trúc trong việc đảm bảo hiệu suất, khả năng mở rộng và tính ổn định của ứng dụng. Các yếu tố như tính đồng nhất, tính trong suốt và tính khả chuyển sẽ được trình bày như những tiêu chí thiết kế quan trọng. Bên cạnh đó, những mô hình kiến trúc phổ biến và công nghệ hỗ trợ như microservices, containerization, cùng các công cụ DevOps sẽ giúp hình dung rõ hơn về cách xây dựng một hệ thống di động hiện đại và hiệu quả.
\end{flushleft}
    % 1.1.
    \subsection{Tổng quan về kiến trúc hệ thống trong ứng dụng di động}
    \renewcommand{\labelitemi}{--}    
    \begin{flushleft}
            \hspace*{0.8cm}Kiến trúc hệ thống trong phát triển ứng dụng di động là nền tảng kỹ thuật bao gồm ba thành phần chính: phần cứng thiết bị, ứng dụng di động và hệ thống server hỗ trợ phía sau. Mục tiêu chính của kiến trúc này là thiết lập một hệ thống có khả năng vận hành đồng nhất, đảm bảo tính trong suốt giữa các thành phần và đạt được mức độ khả chuyển cao.
    \end{flushleft}

    \begin{flushleft}
        \hspace*{0.8cm}Việc xây dựng kiến trúc hệ thống hợp lý đóng vai trò quan trọng đối với sự thành công của một ứng dụng. Trước hết, kiến trúc hệ thống ảnh hưởng trực tiếp đến hiệu suất hoạt động cũng như khả năng mở rộng trong tương lai. Bên cạnh đó, một kiến trúc ổn định sẽ giúp ứng dụng hoạt động mượt mà và hạn chế lỗi phát sinh trong quá trình sử dụng. Quan trọng hơn, khi các thành phần trong hệ thống được tổ chức rõ ràng, việc phát hiện và xử lý lỗi sẽ trở nên nhanh chóng và chính xác hơn.
    \end{flushleft}

    % 1.2.
    \subsection{Các yếu tố quan trọng trong kiến trúc hệ thống}
    \renewcommand{\labelitemi}{--}
    \begin{flushleft}
        \hspace*{0.8cm}Một kiến trúc hệ thống hiệu quả cần đảm bảo ba yếu tố cốt lõi: tính đồng nhất, tính trong suốt và tính khả chuyển.
    \end{flushleft}

    \begin{flushleft}
        \hspace*{0.8cm}Tính đồng nhất là yêu cầu đầu tiên đối với một hệ thống hiện đại. Để đảm bảo sự nhất quán về dữ liệu giữa client và server, các kiến trúc sư phần mềm thường sử dụng các giao thức và tiêu chuẩn truyền thông như RESTful API hoặc GraphQL. Các chuẩn này không chỉ hỗ trợ chuẩn hóa quá trình giao tiếp giữa các thành phần mà còn giúp quản lý dữ liệu được lưu trữ và đồng bộ hóa một cách chính xác và hiệu quả trong toàn hệ thống.
      \end{flushleft}

      \begin{flushleft}
        \hspace*{0.8cm}Tính trong suốt đóng vai trò thiết yếu trong việc phát hiện và xử lý lỗi. Một hệ thống được thiết kế tốt cần có khả năng ghi nhận và thông báo lỗi một cách rõ ràng. Để đạt được điều này, các công cụ như Firebase Crashlytics hoặc Sentry thường được tích hợp vào quá trình vận hành nhằm hỗ trợ theo dõi sự cố theo thời gian thực. Đồng thời, kiến trúc hệ thống cũng cần hỗ trợ việc kiểm tra và sửa lỗi một cách nhanh chóng, đảm bảo giảm thiểu gián đoạn trong quá trình vận hành.
      \end{flushleft}

      \begin{flushleft}
        \hspace*{0.8cm}Tính khả chuyển đề cập đến mức độ linh hoạt của hệ thống khi có sự thay đổi về công nghệ hoặc thành phần. Trong kiến trúc hiện đại, việc áp dụng mô hình microservices hoặc modular architecture cho phép chia tách các thành phần độc lập, từ đó giúp việc thay thế hay nâng cấp trở nên dễ dàng mà không gây ảnh hưởng đến toàn hệ thống. Ngoài ra, công nghệ containerization như Docker hoặc Kubernetes cũng được sử dụng phổ biến nhằm tăng khả năng triển khai linh hoạt và đồng nhất giữa các môi trường khác nhau.
      \end{flushleft}

    % 1.3.
    \subsection{Các mô hình kiến trúc phổ biến trong ứng dụng di động}
    \renewcommand{\labelitemi}{--}
    \begin{flushleft}
        \hspace*{0.8cm}Hiện nay có một số kiến trúc phổ biến trong ứng dụng di động bao gồm:
        \setlength{\leftmargini}{1.5cm}
        \begin{itemize}
            \item Kiến trúc Layers (3 thành phần): Kiến trúc này chia ứng dụng thành ba lớp chính: lớp giao diện người dùng (UI), lớp logic nghiệp vụ (Business Logic), và lớp dữ liệu (Data). Mô hình này giúp tách biệt các thành phần của ứng dụng, làm cho ứng dụng dễ bảo trì và mở rộng.
            \item Kiến trúc MVC (Model-View-Controller): Ở kiến trúc này, ứng dụng được phân chia thành ba thành phần chính: Model (dữ liệu và logic), View (giao diện người dùng), và Controller (xử lý sự kiện và giao tiếp giữa Model và View).
            \item Kiến trúc MVVM (Model-View-ViewModel): Tương tự như MVC nhưng có sự bổ sung của ViewModel, giúp quản lý dữ liệu giữa View và Model, tạo sự tách biệt rõ ràng hơn giữa các thành phần trong ứng dụng.
            \item Kiến trúc MVI (Model-View-Intent): Kiến trúc này phân chia mỗi hành động người dùng (Intent) thành các sự kiện, gửi đến View, rồi View trả lại trạng thái mới cho Model, giúp duy trì tính nhất quán và đồng bộ trong ứng dụng.
            \item Kiến trúc Client-Server: Mô hình này chia ứng dụng thành hai phần: Client (ứng dụng di động) và Server (máy chủ). Client gửi yêu cầu đến Server, Server xử lý và trả về kết quả cho Client. Mô hình này thường được sử dụng trong các ứng dụng cần giao tiếp với cơ sở dữ liệu hoặc dịch vụ trực tuyến.
        \end{itemize}     
    \end{flushleft}

    % 1.4.
    \subsection{Công nghệ và công cụ hỗ trợ}
    \renewcommand{\labelitemi}{--}
    \begin{itemize}
        \item[] Backend cho ứng dụng di động:
        \begin{itemize}
            \item Node.js với NestJS: Node.js là môi trường runtime cho JavaScript, cho phép xử lý các tác vụ bất đồng bộ hiệu quả. NestJS là framework dựa trên Node.js, giúp phát triển các ứng dụng server-side bằng cách sử dụng TypeScript, mang lại sự tổ chức và cấu trúc tốt cho dự án.
            \item Django: Một framework Python mạnh mẽ, Django cung cấp các công cụ để xây dựng ứng dụng web an toàn và nhanh chóng. Django thường được sử dụng trong các ứng dụng yêu cầu bảo mật cao và khả năng mở rộng tốt.
            \item Spring Boot: Là framework Java giúp phát triển các ứng dụng backend nhanh chóng và dễ dàng, đặc biệt là các ứng dụng có tính mở rộng cao và yêu cầu quản lý phức tạp.
        \end{itemize}
    
        \item[] Database:
        \begin{itemize}
            \item MySQL: Hệ quản trị cơ sở dữ liệu quan hệ mã nguồn mở phổ biến, MySQL phù hợp với các ứng dụng yêu cầu tính ổn định cao và khả năng mở rộng.
            \item PostgreSQL: Hệ quản trị cơ sở dữ liệu quan hệ mạnh mẽ, hỗ trợ các tính năng phức tạp hơn MySQL, phù hợp cho các ứng dụng yêu cầu tính nhất quán và linh hoạt cao.
            \item Firebase Firestore: Dịch vụ cơ sở dữ liệu NoSQL của Google, cung cấp tính năng đồng bộ thời gian thực, phù hợp cho các ứng dụng di động yêu cầu đồng bộ dữ liệu nhanh chóng và liên tục.
        \end{itemize}
    
        \item[] Frontend và Mobile Frameworks:
        \begin{itemize}
            \item React Native: Framework giúp phát triển ứng dụng di động cross-platform sử dụng JavaScript và React, tiết kiệm thời gian phát triển với mã nguồn chung cho cả iOS và Android.
            \item Flutter: Framework của Google sử dụng ngôn ngữ Dart, giúp xây dựng giao diện người dùng đẹp mắt và mượt mà cho các ứng dụng di động.
            \item Swift: Ngôn ngữ lập trình dành riêng cho iOS, tối ưu hóa cho các ứng dụng di động trên hệ sinh thái Apple.
            \item Kotlin: Ngôn ngữ chính để phát triển ứng dụng Android, mang lại cú pháp đơn giản và dễ duy trì, tương thích tốt với Java.
        \end{itemize}
    
        \item[] CI/CD và DevOps trong phát triển hệ thống:
        \begin{itemize}
            \item GitHub Actions: Công cụ tự động hóa tích hợp với GitHub, giúp quản lý quy trình CI/CD cho các dự án phần mềm.
            \item Jenkins: Công cụ tự động hóa mã nguồn mở giúp quản lý các quy trình tích hợp và triển khai liên tục.
            \item Docker: Công cụ giúp đóng gói và triển khai ứng dụng trong các container, đảm bảo tính nhất quán giữa các môi trường khác nhau.
        \end{itemize}
    \end{itemize}
   