
\section{Tổng quan về kiến trúc hệ thống}
    % 1.1.
    \subsection{Tổng quan về kiến trúc hệ thống trong ứng dụng di động}
    \renewcommand{\labelitemi}{--}    
    \begin{flushleft}
            \hspace*{0.8cm}Khái niệm kiến trúc hệ thống: Kiến trúc hệ thống trong phát triển ứng dụng di động bao gồm nhiều thành phần như phần cứng thiết bị, ứng dụng di động và hệ thống server hỗ trợ. Mục tiêu của kiến trúc là tạo ra một hệ thống đồng nhất, trong suốt và khả chuyển.
    \end{flushleft}

    \begin{flushleft}
        \hspace*{0.8cm}Cần quan tâm đến kiến trúc hệ thống vì kiến trúc hệ thống:
        \setlength{\leftmargini}{1.5cm}
        \begin{itemize}
            \item Ảnh hưởng trực tiếp đến hiệu suất và khả năng mở rộng của ứng dụng.
            \item Giúp ứng dụng hoạt động ổn định, hạn chế lỗi phát sinh.
            \item Dễ dàng phát hiện và xử lý lỗi giữa các thành phần.
        \end{itemize}
    \end{flushleft}

    % 1.2.
    \subsection{Các yếu tố quan trọng trong kiến trúc hệ thống}
    \renewcommand{\labelitemi}{--}
    \begin{flushleft}
        \hspace*{0.8cm}Tính đồng nhất:
        \setlength{\leftmargini}{1.5cm}
        \begin{itemize}
            \item Dữ liệu cần có sự nhất quán giữa client và server.
            \item Sử dụng RESTful API hoặc GraphQL để chuẩn hóa giao tiếp giữa các thành phần.
            \item Các hệ thống cần đảm bảo dữ liệu được lưu trữ và đồng bộ hóa đúng cách.
        \end{itemize}

        \hspace*{0.8cm}Tính trong suốt:
        \setlength{\leftmargini}{1.5cm}
        \begin{itemize}
            \item Các thành phần trong hệ thống cần có khả năng phát hiện lỗi và thông báo rõ ràng.
            \item Sử dụng các công cụ logging, monitoring như Firebase Crashlytics, Sentry để theo dõi lỗi.
            \item Cấu trúc hệ thống phải cho phép dễ dàng kiểm tra và sửa lỗi nhanh chóng.
        \end{itemize}

        \hspace*{0.8cm}Tính khả chuyển:
        \setlength{\leftmargini}{1.5cm}
        \begin{itemize}
            \item Hệ thống cần có kiến trúc linh hoạt để dễ dàng thay đổi thành phần mà không làm gián đoạn hoạt động.
            \item Sử dụng microservices hoặc modular architecture để tách biệt các thành phần.
            \item Ứng dụng công nghệ containerization như Docker, Kubernetes để tăng tính khả chuyển.
        \end{itemize}
    \end{flushleft}

    % 1.3.
    \subsection{Các mô hình kiến trúc phổ biến trong ứng dụng di động}
    \renewcommand{\labelitemi}{--}
    \begin{flushleft}
        \hspace*{0.8cm}Hiện nay có một số kiến trúc phổ biến trong ứng dụng di động bao gồm:
        \setlength{\leftmargini}{1.5cm}
        \begin{itemize}
            \item Kiến trúc Layers (3 thành phần): Kiến trúc này chia ứng dụng thành ba lớp chính: lớp giao diện người dùng (UI), lớp logic nghiệp vụ (Business Logic), và lớp dữ liệu (Data). Mô hình này giúp tách biệt các thành phần của ứng dụng, làm cho ứng dụng dễ bảo trì và mở rộng.
            \item Kiến trúc MVC (Model-View-Controller): Ở kiến trúc này, ứng dụng được phân chia thành ba thành phần chính: Model (dữ liệu và logic), View (giao diện người dùng), và Controller (xử lý sự kiện và giao tiếp giữa Model và View).
            \item Kiến trúc MVVM (Model-View-ViewModel): Tương tự như MVC nhưng có sự bổ sung của ViewModel, giúp quản lý dữ liệu giữa View và Model, tạo sự tách biệt rõ ràng hơn giữa các thành phần trong ứng dụng.
            \item Kiến trúc MVI (Model-View-Intent): Kiến trúc này phân chia mỗi hành động người dùng (Intent) thành các sự kiện, gửi đến View, rồi View trả lại trạng thái mới cho Model, giúp duy trì tính nhất quán và đồng bộ trong ứng dụng.
            \item Kiến trúc Client-Server: Mô hình này chia ứng dụng thành hai phần: Client (ứng dụng di động) và Server (máy chủ). Client gửi yêu cầu đến Server, Server xử lý và trả về kết quả cho Client. Mô hình này thường được sử dụng trong các ứng dụng cần giao tiếp với cơ sở dữ liệu hoặc dịch vụ trực tuyến.
        \end{itemize}     
    \end{flushleft}

    % 1.4.
    \subsection{Công nghệ và công cụ hỗ trợ}
    \renewcommand{\labelitemi}{--}
    \begin{itemize}
        \item[] Backend cho ứng dụng di động:
        \begin{itemize}
            \item Node.js với NestJS: Node.js là môi trường runtime cho JavaScript, cho phép xử lý các tác vụ bất đồng bộ hiệu quả. NestJS là framework dựa trên Node.js, giúp phát triển các ứng dụng server-side bằng cách sử dụng TypeScript, mang lại sự tổ chức và cấu trúc tốt cho dự án.
            \item Django: Một framework Python mạnh mẽ, Django cung cấp các công cụ để xây dựng ứng dụng web an toàn và nhanh chóng. Django thường được sử dụng trong các ứng dụng yêu cầu bảo mật cao và khả năng mở rộng tốt.
            \item Spring Boot: Là framework Java giúp phát triển các ứng dụng backend nhanh chóng và dễ dàng, đặc biệt là các ứng dụng có tính mở rộng cao và yêu cầu quản lý phức tạp.
        \end{itemize}
    
        \item[] Database:
        \begin{itemize}
            \item MySQL: Hệ quản trị cơ sở dữ liệu quan hệ mã nguồn mở phổ biến, MySQL phù hợp với các ứng dụng yêu cầu tính ổn định cao và khả năng mở rộng.
            \item PostgreSQL: Hệ quản trị cơ sở dữ liệu quan hệ mạnh mẽ, hỗ trợ các tính năng phức tạp hơn MySQL, phù hợp cho các ứng dụng yêu cầu tính nhất quán và linh hoạt cao.
            \item Firebase Firestore: Dịch vụ cơ sở dữ liệu NoSQL của Google, cung cấp tính năng đồng bộ thời gian thực, phù hợp cho các ứng dụng di động yêu cầu đồng bộ dữ liệu nhanh chóng và liên tục.
        \end{itemize}
    
        \item[] Frontend và Mobile Frameworks:
        \begin{itemize}
            \item React Native: Framework giúp phát triển ứng dụng di động cross-platform sử dụng JavaScript và React, tiết kiệm thời gian phát triển với mã nguồn chung cho cả iOS và Android.
            \item Flutter: Framework của Google sử dụng ngôn ngữ Dart, giúp xây dựng giao diện người dùng đẹp mắt và mượt mà cho các ứng dụng di động.
            \item Swift: Ngôn ngữ lập trình dành riêng cho iOS, tối ưu hóa cho các ứng dụng di động trên hệ sinh thái Apple.
            \item Kotlin: Ngôn ngữ chính để phát triển ứng dụng Android, mang lại cú pháp đơn giản và dễ duy trì, tương thích tốt với Java.
        \end{itemize}
    
        \item[] CI/CD và DevOps trong phát triển hệ thống:
        \begin{itemize}
            \item GitHub Actions: Công cụ tự động hóa tích hợp với GitHub, giúp quản lý quy trình CI/CD cho các dự án phần mềm.
            \item Jenkins: Công cụ tự động hóa mã nguồn mở giúp quản lý các quy trình tích hợp và triển khai liên tục.
            \item Docker: Công cụ giúp đóng gói và triển khai ứng dụng trong các container, đảm bảo tính nhất quán giữa các môi trường khác nhau.
        \end{itemize}
    \end{itemize}
   