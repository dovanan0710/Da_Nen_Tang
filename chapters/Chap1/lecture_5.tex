\section{Các yếu tố ảnh hưởng đến chi phí phát triển ứng dụng}

% 5.1
\subsection{Các yếu tố chính:}
\renewcommand{\labelitemi}{--}    
    \begin{flushleft}
        \hspace*{0.8cm}Features and Functionality: Tính năng càng phức tạp, chi phí càng cao.
    \end{flushleft}

    \begin{flushleft}
      \hspace*{0.8cm}App Infrastructure: Liên quan đến cách tổ chức hệ thống backend, API, hệ thống lưu trữ, bảo mật, quản lý hiệu năng,...
    \end{flushleft}

    \begin{flushleft}
      \hspace*{0.8cm}App Complexity: Mức độ phức tạp về luồng xử lý, logic kinh doanh.
    \end{flushleft}

    \begin{flushleft}
      \hspace*{0.8cm}UX/UI Design: Thiết kế giao diện thân thiện, hiện đại cần đầu tư nhiều hơn.
    \end{flushleft}

    \begin{flushleft}
      \hspace*{0.8cm}App Maintenance: Chi phí bảo trì sau khi triển khai.
    \end{flushleft}

    \begin{flushleft}
      \hspace*{0.8cm}App Security: Yêu cầu bảo mật cao làm tăng chi phí.
    \end{flushleft}

    \begin{flushleft}
      \hspace*{0.8cm}App Category: Tùy theo ứng dụng thuộc lĩnh vực nào (y tế, tài chính, mạng xã hội...) sẽ có độ phức tạp khác nhau.
    \end{flushleft}

    \begin{flushleft}
      \hspace*{0.8cm}\# of Platforms \& Pages: Càng nhiều nền tảng (iOS/Android/Web), giao diện càng đa dạng → chi phí tăng.
    \end{flushleft}

    \begin{flushleft}
      \hspace*{0.8cm}Location of the Development Team: Địa điểm nhóm phát triển quyết định đến chi phí nhân công.
    \end{flushleft}

    \begin{flushleft}
      \hspace*{0.8cm}$\Rightarrow$ Vai trò then chốt của kiến trúc phần mềm - App Infrastructure (Hạ tầng ứng dụng):
      \setlength{\leftmargini}{1.5cm}
      \begin{itemize}
          \item Là xương sống của toàn bộ hệ thống phần mềm, quyết định khả năng mở rộng, hiệu suất, bảo mật, tích hợp dịch vụ khác (API, cơ sở dữ liệu, cloud...).
          \item Việc thiết kế hạ tầng tốt từ đầu giúp giảm chi phí bảo trì về sau và tăng độ ổn định của ứng dụng.
      \end{itemize}
    \end{flushleft}

    \begin{flushleft}
      \hspace*{0.8cm}
    \end{flushleft}

    \begin{flushleft}
      \hspace*{0.8cm}
    \end{flushleft}

% 
\subsection{Chi phí thuê nhân sự theo khu vực địa lý:}
\renewcommand{\labelitemi}{--}    
    \begin{flushleft}
        \hspace*{0.8cm}Bảng trên thể hiện khoản chi phí thuê theo giờ cho các vai trò chuyên môn trong ngành công nghệ thông tin, dựa trên thống kê từ 4 khu vực là United States (Hoa Kỳ), Latin America (Mỹ Latin), Eastern Europe (Đông Âu) và Asia (Châu Á). Đồng thời bao gồm 12 vai trò phổ biến trong các dự án phần mềm.
    \end{flushleft}

    \begin{flushleft}
      \hspace*{0.8cm}Nhóm Quản lý và Thiết kế Kiến trúc:
      \setlength{\leftmargini}{1.5cm}
      \begin{itemize}
          \item Nhóm này bao gồm 3 vai trò Business Analyst, Architect và Project Manager.
          \item Đây là nhóm có mức phí cao nhất, đặc biệt tại Hoa Kỳ (lên đến gần \$300/giờ).
          \item Mức giá giảm dần theo khu vực: US > Latin America > Eastern Europe > Asia
          \item Châu Á có mức phí thấp nhất, từ \$30 – \$48/giờ, khiến khu vực này hấp dẫn với các công ty thuê ngoài (outsourcing).
          \item[]$\Rightarrow$ Đây là các vị trí quyết định kiến trúc phần mềm, quy trình quản lý và kết nối giữa yêu cầu kinh doanh với kỹ thuật, nên đòi hỏi kinh nghiệm và kỹ năng cao.
      \end{itemize}
    \end{flushleft}

    \begin{flushleft}
      \hspace*{0.8cm}Nhóm Lập trình viên (Developer):
      \setlength{\leftmargini}{1.5cm}
      \begin{itemize}
          \item Nhóm này được chia theo kinh nghiệm của lập trình viên, bao gồm Jr. Developer, Mid-Level Dev, Sr. Developer và Lead Developer
          \item Mức lương tăng dần theo cấp bậc từ Jr. → Mid → Sr. → Lead.
          \item Sr. Developer và Lead Developer có chi phí gần tiệm cận nhóm kiến trúc sư tại một số khu vực.
          \item Asia tiếp tục là nơi có mức chi phí thấp nhất ở mọi cấp độ.
      \end{itemize}
    \end{flushleft}

    \begin{flushleft}
      \hspace*{0.8cm}Nhóm Kiểm thử phần mềm (QA):
      \setlength{\leftmargini}{1.5cm}
      \begin{itemize}
          \item Nhóm này cũng được chia dựa trên kinh nghiệm, bao gồm Junior QA, Mid-Level QA và Senior QA.
          \item QA ít tốn kém hơn Developer hoặc Architect, nhưng vẫn có mức chênh lệch đáng kể giữa các khu vực.
          \item Sr. QA tại Mỹ có thể lên đến \$169/giờ, cao hơn cả Mid-Level Dev tại Mỹ (\$140).
      \end{itemize}
    \end{flushleft}

    \begin{flushleft}
      \hspace*{0.8cm}Thiết kế giao diện (Graphic Designer):
      \setlength{\leftmargini}{1.5cm}
      \begin{itemize}
          \item Vai trò thiết kế tuy không chuyên sâu kỹ thuật nhưng vẫn chiếm chi phí tương đối, đặc biệt ở các thị trường phát triển như Mỹ.
      \end{itemize}
    \end{flushleft}