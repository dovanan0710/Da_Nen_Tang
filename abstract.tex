\chapter*{Tóm tắt tài liệu}
\addcontentsline{toc}{chapter}{Tóm tắt tài liệu}

Các thiết bị di động đã trở thành một phần không thể thiếu trong đời sống hiện đại cũng như trong lĩnh vực phát triển phần mềm — đó là minh chứng rõ ràng cho vai trò trung tâm của kiến trúc di động trong việc xây dựng và triển khai ứng dụng. Với sự phát triển nhanh chóng của phần cứng thiết bị và hệ điều hành, kiến trúc di động ngày càng đa dạng, bao gồm các mô hình native, hybrid và cross-platform. Mỗi mô hình đều có những ưu điểm và thách thức riêng, đặt ra yêu cầu cho nhà phát triển phải cân nhắc kỹ lưỡng trong việc lựa chọn kiến trúc phù hợp với đặc thù của từng dự án. Nhờ hiểu biết sâu sắc về kiến trúc di động, quá trình phát triển ứng dụng không chỉ trở nên hiệu quả hơn mà còn mang lại hiệu suất cao và trải nghiệm người dùng tối ưu.
