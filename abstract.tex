\documentclass[12pt]{article}
\usepackage[utf8]{inputenc}
\usepackage[vietnamese]{babel}
\usepackage[hidelinks]{hyperref}
\usepackage[a4paper, margin=2.5cm]{geometry}
\usepackage{setspace}
\setstretch{1.5}

\begin{document}
\begin{abstract}
Trong thời đại công nghệ số, các thiết bị di động đã trở thành một phần không thể thiếu trong đời sống hàng ngày cũng như trong lĩnh vực phát triển phần mềm. Kiến trúc di động đóng vai trò quan trọng trong việc định hình cách xây dựng và triển khai các ứng dụng di động hiện đại. Bài viết này trình bày một cái nhìn tổng quan về kiến trúc di động, bao gồm sự phát triển của phần cứng thiết bị, hệ điều hành di động, và các mô hình kiến trúc phổ biến như kiến trúc native, hybrid và cross-platform. Ngoài ra, bài viết cũng phân tích ưu điểm và thách thức của từng mô hình, từ đó giúp các nhà phát triển lựa chọn hướng đi phù hợp cho từng dự án cụ thể. Việc hiểu rõ kiến trúc di động không chỉ hỗ trợ quá trình phát triển ứng dụng hiệu quả hơn mà còn giúp tối ưu hóa hiệu suất và trải nghiệm người dùng.
\end{abstract}

\end{document}
